\begin{table}
    \centering
    \caption[Municipios con Mayor y Menor Grado de Pobreza por Ingresos]{Municipios con Mayor y Menor Grado de Pobreza\newline por Ingresos}
    \label{tbl:CONEVAL:MunicipiosPobreza}
    \footnotesize
    \begin{tabular}{c|l|r||c|c|c}
                       &                &           & POBREZA     & POBREZA DE  & POBREZA DE \\
                       & MUNICIPIO      & POBLACIÓN & ALIMENTARIA & CAPACIDADES & PATRIMONIO \\ 
        \hline \hline
        Estatal        & Guanajuato     & 4,893,812 & 18.9 \%     & 26.6 \%     & 51.6 \%    \\ 
        \hline
        Municipios     & Xichú          &    10,592 & 60.8 \%     & 68.5 \%     & 83.6 \%    \\ 
        con mayor      & Atarjea        &     5,035 & 58.8 \%     & 67.1 \%     & 83.2 \%    \\ 
        porcentaje     & Tierra Blanca  &    16,136 & 50.9 \%     & 59.4 \%     & 77.3 \%    \\ 
        de pobreza     & Victoria       &    19,112 & 47.4 \%     & 55.4 \%     & 73.2 \%    \\ 
                       & Santa Catarina &     4,544 & 41.3 \%     & 47.8 \%     & 64.4 \%    \\ 
        \hline
        Municipios     & Irapuato       &   463,103 & 14.0 \%     & 21.2 \%     & 46.9 \%    \\ 
        con menor      & Uriangato      &    53,077 & 11.6 \%     & 19.9 \%     & 50.4 \%    \\ 
        porcentaje     & Celaya         &   415,869 & 11.5 \%     & 17.9 \%     & 41.3 \%    \\ 
        de pobreza     & Moroleón       &    46,751 &  9.5 \%     & 16.1 \%     & 42.6 \%    \\ 
                       & León           & 1,278,087 &  7.9 \%     & 13.6 \%     & 38.2 \%    \\ 
        \hline
        \multicolumn{6}{l}{\footnotesize Fuente: \citep{Coneval2009}}
    \end{tabular}
\end{table}
