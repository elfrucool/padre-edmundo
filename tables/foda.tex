\begin{table}[h!]
    \centering
    \caption{Análisis FODA}
    \label{tbl:Foda}
    \footnotesize
    \begin{tabular}{r|p{5in}}
    	\multicolumn{2}{c}{FORTALEZAS} \\
    	\hline
    	\hline
    	F-1 & Se cuenta con un inmueble con 40 salones, dos espacios para talleres, aulas para laboratorios, áreas administrativas, auditorio, campo de futbol, etc. \\
    	F-2 & Experiencia exitosa (SLP, Saltillo) \\
    	F-3 & Prestigio de la orden Salesianos de Don Bosco \\
    	F-4 & Interés por parte de algunos padres de la inspectoría por implementar el bachillerato tecnológico \\
    	F-5 & Un grupo de alrededor de 70 ex salesianos interesados y colaborando en el proyecto \\
    	F-6 & Ya se cuenta con las incorporaciones correspondientes ante la SEP \\
    	\hline
    	\multicolumn{2}{c}{DEBILIDADES} \\
    	\hline
    	\hline
    	D-1 & El P. Inspector aún no se decide por el proyecto \\
    	D-2 & Son pocos padres salesianos en Irapuato (menos de 6) \\
    	D-3 & La comunidad (sacerdotes, religiosas, laicos) tiene múltiples responsabilidades además de la escuela \\
    	D-4 & El \emph{staff} del proyecto no est\'a plenamente consolidado \\
    	\hline
    	\multicolumn{2}{c}{OPORTUNIDADES} \\
    	\hline
    	\hline
    	O-1 & Hay un fuerte impulso al desarrollo del corredor industrial de Guanajuato \\
    	O-2 & Por su posición geográfica, Irapuato, se está constituyendo en el centro de una conformación metropolitana, en la que para empezar se está integrando Salamanca \\
    	O-3 & Irapuato se puede convertir en un polo educativo, que dé servicio a varios municipios \\
    	O-4 & La cercanía con Querétaro coloca a Irapuato en situación vinculante Guanajuato-Querétaro \\
    	O-5 & El gobierno del estado está impulsando fuerte la educación media superior y superior \\
    	O-6 & Empresas grandes y fuertes están invirtiendo en la entidad \\
    	O-7 & El interés por los bachilleratos tecnológicos está creciendo \\
    	\hline
    	\multicolumn{2}{c}{AMENAZAS} \\
    	\hline
    	\hline
    	A-1 & La crisis económica actual podría obligar a los estudiantes a inscribirse a una escuela oficial o a abandonar los estudios \\
    	A-2 & Mercado competido dificulta la entrada de nuevos competidores \\
    	A-3 & La Universidad Quetzalcóatl ofrece más carreras técnicas y su precio general es más económico que el propuesto \\
    	A-4 & Existe el riesgo de no estar listos a tiempo para promover la escuela (febrero-abril) y que todo se retrase un año \\
    	\hline
    	\multicolumn{2}{l}{\footnotesize Fuente: Elaboración propia.} \\
    \end{tabular}
\end{table}

