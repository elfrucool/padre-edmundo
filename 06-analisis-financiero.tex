\chapter{Análisis Financiero}

En este capítulo se analiza financieramente el proyecto con el fin de identificar características que lo hagan más atractivo a posibles patrocinadores y entidades financieras que otorguen préstamos. Está dividido en tres partes, en la primera (análisis estructural o vertical) se analizará la estructura intemporal del estado de resultados, en la segunda (análisis intertemporal u horizontal) se revisará su evolución en el tiempo y en la tercera (razones financieras) se extraerán variables que describan el proyecto.

Tanto el análisis vertical como el horizontal pueden realizarse igualmente sobre el balance general, sin embargo, para entidades crediticias y patrocinadores son mucho más relevantes estos análisis sobre el estado de resultados proforma\footnote{Para el análisis vertical y el horizontal se realizó una redacción propia a partir de las siguientes fuentes: \citep{brock1987contabilidad, mejia2006diccionario, dobarganes2005contabilidad}}.

\section{Análisis estructural o vertical}
\label{sec:AnalisisVertical}

El análisis estructural o vertical identifica la estructura del estado de resultados en términos de porcentajes sobre los ingresos netos. Permite identificar qué aspecto o rubro impacta con mayor fuerza en la reducción de las utilidades, así como la relación entre utilidades e ingresos.

Observaciones sobre el análisis vertical (cuadros \ref{tbl:Vertical:1} y \ref{tbl:Vertical:2}):

\begin{itemize}
	\item La utilidad bruta se mantiene proporcionalmente casi constante con variaciones menores al 1\% y tendencia decreciente hacia el final (98.91\%--98.57\%).
	\item La proporción del margen de utilidad se incrementa notablemente los primeros años (de 37.75\% a 46.45\% y luego a 50.52\%) y tiende a estabilizarse hacia el final (de 46.09\% a 45.62\%), esto a pesar del incremento al porcentaje de becas.
	\item La utilidad de operación pasó de -24.45\% a 15.67\% en el año 2, posteriormente sufrió una disminución a 15.14\% y finalmente repuntó a 18.51\%. Entre las posibles causas se encuentran el incremento a las becas y la reestructuración del organigrama.
	\item Al carecer de amortizaciones al activo diferido, la utilidad antes de impuestos es exactamente la misma que la utilidad de operación.
	\item La utilidad de la operación pasa de -26.45\% el primer año a 9.48\% el último, en incremento constante con excepción del cuarto año.
	\item Se confirma que el costo más importante viene dado por los sueldos, representando entre el 57.98\% y el 66.37\% (a excepción del primer año que corresponde al 80.37\%)
\end{itemize}

La estructura de costos está dentro del rango de la industria. Por citar un ejemplo, la Universidad La Salle de Pachuca gasta 75\% del ingreso en el pago de nómina\footnote{Entrevista con Héctor Miranda, exdirector del bachillerato}.

Se observa una tendencia de los indicadores a estabilizarse a medida que avanza el tiempo, esto a pesar de variaciones en el organigrama y porcentaje de becas, lo mismo que el efecto de la inflación sobre una colegiatura que no se incrementa durante esos mismos cinco años.

\begin{table}
    \caption{Análisis Vertical}
    \label{tbl:Vertical:1}
    \centering
    \scriptsize
    \begin{tabular}{@{\hspace{1mm}}r@{\hspace{1mm}}|@{\hspace{1mm}}l@{\hspace{1mm}}*{6}{|@{\hspace{1mm}}r@{\hspace{1mm}}}}
	& \multicolumn{1}{c|}{} &
	    \multicolumn{6}{c}{AÑO} \\
	\cline{3-8}
	& \multicolumn{1}{c|}{CONCEPTO} &
	    \multicolumn{1}{c|}{1} &
	    \multicolumn{1}{c|}{\%} &
	    \multicolumn{1}{c|}{2} &
	    \multicolumn{1}{c|}{\%} &
	    \multicolumn{1}{c|}{3} &
	    \multicolumn{1}{c}{\%} \\
    %--------------------------------------------------
	\hline
	\hline
	1	&	INGRESOS NETOS                   &  5,602,800.00 	&	100.00	&	 11,205,600.00 	&	100.00	&	 16,808,400.00 	&	100.00 \\
	2	&	\multicolumn{7}{l}{COSTOS VARIABLES}                \\
	\hline
	3	&	Papelería                        &  12,372.00 	&	0.22	&	 23,373.00 	&	0.21	&	 35,840.07 	&	0.21 \\
	4	&	Material Didáctico               &  13,440.00 	&	0.24	&	 26,460.00 	&	0.24	&	 42,600.60 	&	0.25 \\
	5	&	Material talleres                &  35,000.00 	&	0.62	&	 73,500.00 	&	0.66	&	 99,225.00 	&	0.59 \\
	\hline
	6	&	TOTAL COSTOS VARIABLES           &  60,812.00 	&	1.09	&	 123,333.00 	&	1.10	&	 177,665.67 	&	1.06 \\
	\hline
	7	&	UTILIDAD BRUTA                   &  5,541,988.00 	&	98.91	&	 11,082,267.00 	&	98.90	&	 16,630,734.33 	&	98.94 \\
	\hline
	\hline
	8	&	\multicolumn{7}{l}{COSTOS FIJOS}                    \\
	\hline
	9	&	Servicios subcontratados         &  836,000.00 	&	14.92	&	 1,570,800.00 	&	14.02	&	 1,649,340.00 	&	9.81 \\
	10	&	Servicios (agua, luz, teléfono)  &  520,000.00 	&	9.28	&	 546,000.00 	&	4.87	&	 573,300.00 	&	3.41 \\
	11	&	Sueldos Directos                 &  2,070,750.24 	&	36.96	&	 3,760,368.30 	&	33.56	&	 5,915,755.70 	&	35.20 \\
	\hline
	12	&	TOTAL COSTOS FIJOS               &  3,426,750.24 	&	61.16	&	 5,877,168.30 	&	52.45	&	 8,138,395.70 	&	48.42 \\
	\hline
	13	&	MARGEN DE UTILIDAD               &  2,115,237.76 	&	37.75	&	 5,205,098.70 	&	46.45	&	 8,492,338.63 	&	50.52 \\
	\hline
	\hline
	14	&	\multicolumn{7}{l}{GASTOS DE OPERACIÓN}             \\
	\hline
	15	&	\multicolumn{7}{l}{GASTOS DE ADMINISTRACIÓN}        \\
	\hline
	16	&	Sueldos Indirectos               &  1,195,708.80 	&	21.34	&	 1,225,601.52 	&	10.94	&	 2,284,075.56 	&	13.59 \\
	17	&	Sueldos Administrativos          &  1,236,471.60 	&	22.07	&	 1,511,110.97 	&	13.49	&	 2,569,585.01 	&	15.29 \\
	18	&	Depreciación anual               &  560,415.65 	&	10.00	&	 580,415.65 	&	5.18	&	 610,804.65 	&	3.63 \\
	\hline
	19	&	\multicolumn{7}{l}{GASTOS DE VENTA}                 \\
	\hline
	20	&	Publicidad                       &  180,000.00 	&	3.21	&	 189,000.00 	&	1.69	&	 198,450.00 	&	1.18 \\
	\hline
	21	&	TOTAL DE GASTOS DE OPERACIÓN     &  3,172,596.05 	&	56.63	&	 3,506,128.13 	&	31.29	&	 5,662,915.21 	&	33.69 \\
	\hline
	22	&	\multicolumn{7}{l}{GASTOS Y PRODUCTOS FINANCIEROS}  \\
	\hline
	23	&	Gastos Financieros               &  312,644.30 	&	5.58	&	 257,491.49 	&	2.30	&	 195,343.92 	&	1.16 \\
	\hline
	24	&	TOTAL GASTOS FINANCIEROS         &  312,644.30 	&	5.58	&	 257,491.49 	&	2.30	&	 195,343.92 	&	1.16 \\
	\hline
	25	&	UTILIDAD DE OPERACIÓN            & -1,370,002.59 	&	-24.45	&	 1,441,479.08 	&	12.86	&	 2,634,079.50 	&	15.67 \\
	\hline
	\hline
	26	&	\multicolumn{7}{l}{OTROS GASTOS Y PRODUCTOS}        \\
	\hline
	27	&	Amortización del Activo Diferido &  0.0 	&		&	 0.0 	&		&	 0.0 	&	 \\
	\hline
	28	&	UTILIDAD ANTES DE IMPUESTOS      & -1,370,002.59 	&	-24.45	&	 1,441,479.08 	&	12.86	&	 2,634,079.50 	&	15.67 \\
	\hline
	\hline
	29	&	Provisión de ISR$^{/a}$          &  0.0 	&	0.00	&	 403,614.14 	&	3.60	&	 737,542.26 	&	4.39 \\
	30	&	Provisión de PTU$^{/a}$          &  0.0 	&	0.00	&	 144,147.91 	&	1.29	&	 263,407.95 	&	1.57 \\
	32	&	Impuesto al Depósito en Efectivo &  112,056.00 	&	2.00	&	 224,112.00 	&	2.00	&	 336,168.00 	&	2.00 \\
	\hline
	33	&	TOTAL IMPUESTOS Y PRESTACIONES   &  112,056.00 	&	2.00	&	 771,874.05 	&	6.89	&	 1,337,118.21 	&	7.96 \\
	\hline
	\hline
	34	&	UTILIDAD DEL EJERCICIO           &  -1,482,058.59 	&	-26.45	&	 669,605.03 	&	5.98	&	 1,296,961.29 	&	7.72 \\
	%--------------------------------------------------
	\hline
	\multicolumn{8}{l}{\footnotesize Fuente: Elaboración Propia, 2010.} \\
    \multicolumn{8}{p{5.5in}}{$^{/a}$ \footnotesize A pesar de que la figura legal es una I.A.P. se consideró la provisión de ISR y PTU con la finalidad de \emph{castigar} el proyecto y de esta forma minimizar el riesgo de pérdidas.}
    \end{tabular}
\end{table}


\begin{table}
    \caption{Análisis Vertical (continuación)}
    \label{tbl:Vertical:2}
    \centering
    \scriptsize
    \begin{tabular}{@{\hspace{1mm}}r@{\hspace{1mm}}|@{\hspace{1mm}}l@{\hspace{1mm}}*{4}{|@{\hspace{1mm}}r@{\hspace{1mm}}}}
	& \multicolumn{1}{c|}{} &
	    \multicolumn{4}{c}{AÑO} \\
	\cline{3-6}
	& \multicolumn{1}{c|}{CONCEPTO} &
	    \multicolumn{1}{c|}{4} &
	    \multicolumn{1}{c|}{\%} &
	    \multicolumn{1}{c|}{5} &
	    \multicolumn{1}{c}{\%} \\
	%--------------------------------------------------
	\hline
	\hline
	1	&	INGRESOS NETOS                                       & 19,224,000.00 	&	100.00	&	 21,600,000.00 	&	100.00 \\
	\hline
	\hline
	2	&	\multicolumn{5}{l}{COSTOS VARIABLES}                \\
	\hline
	3	&	Papelería                                            & 48,990.69 	&	0.25	&	 57,323.27 	&	0.27 \\
	4	&	Material Didáctico                                   & 58,344.30 	&	0.30	&	 69,429.72 	&	0.32 \\
	5	&	Material talleres                                    & 138,915.00 	&	0.72	&	 182,325.94 	&	0.84 \\
	\hline
	6	&	TOTAL COSTOS VARIABLES                               & 246,249.99 	&	1.28	&	 309,078.93 	&	1.43 \\
	\hline
	7	&	UTILIDAD BRUTA                                       & 18,977,750.01 	&	98.72	&	 21,290,921.07 	&	98.57 \\
	\hline
	\hline
	8	&	\multicolumn{5}{l}{COSTOS FIJOS}                    \\
	9	&	Servicios subcontratados                             & 1,731,807.00 	&	9.01	&	 1,818,397.35 	&	8.42 \\
	10	&	Servicios (agua, luz, telefono)                      & 601,965.00 	&	3.13	&	 632,063.25 	&	2.93 \\
	11	&	Sueldos Directos                                     & 7,784,415.02 	&	40.49	&	 8,986,902.29 	&	41.61 \\
	\hline
	12	&	TOTAL COSTOS FIJOS                                   & 10,118,187.02 	&	52.63	&	 11,437,362.89 	&	52.95 \\
	\hline
	13	&	MARGEN DE UTILIDAD                                   & 8,859,562.99 	&	46.09	&	 9,853,558.19 	&	45.62 \\
	\hline
	\hline
	14	&	\multicolumn{5}{l}{GASTOS DE OPERACIÓN}             \\
	15	&	\multicolumn{5}{l}{GASTOS DE ADMINISTRACIÓN}        \\
	16	&	Sueldos Indirectos                                   & 2,341,177.45 	&	12.18	&	 2,399,706.89 	&	11.11 \\
	17	&	Sueldos Administrativos                              & 2,633,824.63 	&	13.70	&	 2,699,670.25 	&	12.50 \\
	18	&	Depreciación anual                                   & 640,980.69 	&	3.33	&	 490,980.69 	&	2.27 \\
	19	&	\multicolumn{5}{l}{GASTOS DE VENTA}                 \\
	20	&	Publicidad                                           & 208,372.50 	&	1.08	&	 218,791.13 	&	1.01 \\
	\hline
	21	&	TOTAL DE GASTOS DE OPERACIÓN                         & 5,824,355.27 	&	30.30	&	 5,809,148.94 	&	26.89 \\
	\hline
	22	&	\multicolumn{5}{l}{GASTOS Y PRODUCTOS FINANCIEROS}  \\
	23	&	Gastos Financieros                                   & 125,314.48 	&	0.65	&	 46,403.55 	&	0.21 \\
	\hline
	24	&	TOTAL GASTOS FINANCIEROS                             & 125,314.48 	&	0.65	&	 46,403.55 	&	0.21 \\
	\hline
	25	&	UTILIDAD DE OPERACIÓN                                & 2,909,893.25 	&	15.14	&	 3,998,005.69 	&	18.51 \\
	\hline
	\hline
	26	&	\multicolumn{5}{l}{OTROS GASTOS Y PRODUCTOS}        \\
	27	&	Amortización del Activo Diferido                     & 0.0 	&		&	 0.0 	&	 \\
	\hline
	28	&	UTILIDAD ANTES DE IMPUESTOS                          & 2,909,893.25 	&	15.14	&	 3,998,005.69 	&	18.51 \\
	\hline
	\hline
	29	&	Provisión de ISR                                     & 814,770.11 	&	4.24	&	 1,119,441.59 	&	5.18 \\
	30	&	Provisión de PTU                                     & 290,989.32 	&	1.51	&	 399,800.57 	&	1.85 \\
	32	&	Impuesto al Depósido en Efectivo                     & 384,480.00 	&	2.00	&	 432,000.00 	&	2.00 \\
	\hline
	33	&	TOTAL IMPUESTOS Y PRESTACIONES                       & 1,490,239.43 	&	7.75	&	 1,951,242.16 	&	9.03 \\
	\hline
	\hline
	34	&	UTILIDAD DEL EJERCICIO                               & 1,419,653.81 	&	7.38	&	 2,046,763.53 	&	9.48 \\
	%--------------------------------------------------
	\hline
	\multicolumn{6}{l}{\footnotesize Fuente: Elaboración Propia, 2010.}
    \end{tabular}
\end{table}


\section{Análisis intertemporal u horizontal}
\label{sec:AnalisisHorizontal}

El análisis vertical permite observar la evolución en el tiempo del proyecto, se miden incrementos porcentuales de cada uno de los rubros del estado de resultados.

Para el bachillerato, se tienen las siguientes observaciones derivadas del análisis horizontal (cuadros \ref{tbl:Horizontal:1} y \ref{tbl:Horizontal:2}):

\begin{itemize}
	\item Del año 1 al 2 es cuando se dan los cambios más profundos, con incrementos superiores a 140\% en el caso de cantidades positivas y cambios de -200\% para cantidades negativas (de negativo pasa a positivo).
	\item Del año 2 al 3 el incremento más significativo está en la utilidad del ejercicio (93.69\%).
	\item Del año 3 al 4 el ingreso neto se incrementó 14.37\%, mientras que las utilidades se incrementaron 9.46\%.
	\item Del año 4 al 5 las utilidades se incrementaron 44.17\% contra 12.36\% de los ingresos.
	\item Las utilidades se incrementan a un porcentaje siempre mayor a 9\%.
\end{itemize}

La conclusión del análisis horizontal es que a medida que avanza el tiempo, el proyecto tiende a mejorar financieramente.

\begin{table}
    \caption{Análisis Horizontal}
    \label{tbl:Horizontal:1}
    \centering
    \scriptsize
    \begin{tabular}{@{\hspace{1mm}}r@{\hspace{1mm}}|@{\hspace{1mm}}l@{\hspace{1mm}}*{5}{|@{\hspace{1mm}}r@{\hspace{1mm}}}}
	& \multicolumn{1}{c|}{} &
	    \multicolumn{5}{c}{AÑO} \\
	\cline{3-7}
	& \multicolumn{1}{c|}{CONCEPTO} &
	    \multicolumn{1}{c|}{1} &
	    \multicolumn{1}{c|}{\%$\delta(1-2)$} &
	    \multicolumn{1}{c|}{2} &
	    \multicolumn{1}{c|}{\%$\delta(2-3)$} &
	    \multicolumn{1}{c}{3} \\
	%--------------------------------------------------
	\hline
	\hline
	1	&	INGRESOS NETOS                                       &  5,602,800.00 	&	100.00	&	 11,205,600.00 	&	50.00	&	 16,808,400.00  \\
	\hline
	\hline
	2	&	\multicolumn{6}{l}{COSTOS VARIABLES}                \\
	\hline
	3	&	Papelería                                            &  12,372.00 	&	88.92	&	 23,373.00 	&	53.34	&	 35,840.07  \\
	4	&	Material Didáctico                                   &  13,440.00 	&	96.88	&	 26,460.00 	&	61.00	&	 42,600.60  \\
	5	&	Material talleres                                    &  35,000.00 	&	110.00	&	 73,500.00 	&	35.00	&	 99,225.00  \\
	\hline
	6	&	TOTAL COSTOS VARIABLES                               &  60,812.00 	&	102.81	&	 123,333.00 	&	44.05	&	 177,665.67  \\
	\hline
	7	&	UTILIDAD BRUTA                                       &  5,541,988.00 	&	99.97	&	 11,082,267.00 	&	50.07	&	 16,630,734.33  \\
	\hline
	\hline
	8	&	\multicolumn{6}{l}{COSTOS FIJOS}                    \\
	\hline
	9	&	Servicios subcontratados                             &  836,000.00 	&	87.89	&	 1,570,800.00 	&	5.00	&	 1,649,340.00  \\
	10	&	Servicios (agua, luz, telefono)                      &  520,000.00 	&	5.00	&	 546,000.00 	&	5.00	&	 573,300.00  \\
	11	&	Sueldos Directos                                     &  2,070,750.24 	&	81.59	&	 3,760,368.30 	&	57.32	&	 5,915,755.70  \\
	\hline
	12	&	TOTAL COSTOS FIJOS                                   &  3,426,750.24 	&	71.51	&	 5,877,168.30 	&	38.47	&	 8,138,395.70  \\
	\hline
	13	&	MARGEN DE UTILIDAD                                   &  2,115,237.76 	&	146.08	&	 5,205,098.70 	&	63.15	&	 8,492,338.63  \\
	\hline
	\hline
	14	&	\multicolumn{6}{l}{GASTOS DE OPERACIÓN}             \\
	\hline
	15	&	\multicolumn{6}{l}{GASTOS DE ADMINISTRACIÓN}        \\
	\hline
	16	&	Sueldos Indirectos                                   &  1,195,708.80 	&	2.50	&	 1,225,601.52 	&	86.36	&	 2,284,075.56  \\
	17	&	Sueldos Administrativos                              &  1,236,471.60 	&	22.21	&	 1,511,110.97 	&	70.05	&	 2,569,585.01  \\
	18	&	Depreciación anual                                   &  560,415.65 	&	3.57	&	 580,415.65 	&	5.24	&	 610,804.65  \\
	\hline
	19	&	\multicolumn{6}{l}{GASTOS DE VENTA}                 \\
	\hline
	20	&	Publicidad                                           &  180,000.00 	&	5.00	&	 189,000.00 	&	5.00	&	 198,450.00  \\
	\hline
	21	&	TOTAL DE GASTOS DE OPERACIÓN                         &  3,172,596.05 	&	10.51	&	 3,506,128.13 	&	61.51	&	 5,662,915.21  \\
	\hline
	22	&	\multicolumn{6}{l}{GASTOS Y PRODUCTOS FINANCIEROS}  \\
	\hline
	23	&	Gastos Financieros                                   &  312,644.30 	&	-17.64	&	 257,491.49 	&	-24.14	&	 195,343.92  \\
	\hline
	24	&	TOTAL GASTOS FINANCIEROS                             &  312,644.30 	&	-17.64	&	 257,491.49 	&	-24.14	&	 195,343.92  \\
	\hline
	25	&	UTILIDAD DE OPERACIÓN                                & -1,370,002.59 	&	-205.22	&	 1,441,479.08 	&	82.73	&	 2,634,079.50  \\
	\hline
	\hline
	26	&	\multicolumn{6}{l}{OTROS GASTOS Y PRODUCTOS}        \\
	\hline
	27	&	Amortización del Activo Diferido                     &  0.0 	&		&	 0.0 	&		&	 0.0  \\
	\hline
	28	&	UTILIDAD ANTES DE IMPUESTOS                          & -1,370,002.59 	&	-205.22	&	 1,441,479.08 	&	82.73	&	 2,634,079.50  \\
	\hline
	\hline
	29	&	Provisión de ISR                                     &  0.0 	&	N/A	&	 403,614.14 	&	82.73	&	 737,542.26  \\
	30	&	Provisión de PTU                                     &  0.0 	&	N/A	&	 144,147.91 	&	82.73	&	 263,407.95  \\
	32	&	Impuesto al Depósido en Efectivo                     &  112,056.00 	&	100.00	&	 224,112.00 	&	50.00	&	 336,168.00  \\
	\hline
	33	&	TOTAL IMPUESTOS Y PRESTACIONES                       &  112,056.00 	&	588.83	&	 771,874.05 	&	73.23	&	 1,337,118.21  \\
	\hline
	\hline
	34	&	UTILIDAD DEL EJERCICIO                               & -1,482,058.59 	&	-145.18	&	 669,605.03 	&	93.69	&	 1,296,961.29  \\
	%--------------------------------------------------
	\hline
	\multicolumn{7}{l}{\footnotesize Fuente: Elaboración Propia, 2010.}
    \end{tabular}
\end{table}


\begin{table}
    \caption{Análisis Horizontal (Continuación)}
    \label{tbl:Horizontal:2}
    \centering
    \scriptsize
    \begin{tabular}{@{\hspace{1mm}}r@{\hspace{1mm}}|@{\hspace{1mm}}l@{\hspace{1mm}}*{5}{|@{\hspace{1mm}}r@{\hspace{1mm}}}}
	& \multicolumn{1}{c|}{} &
	    \multicolumn{5}{c}{AÑO} \\
	\cline{3-7}
	& \multicolumn{1}{c|}{CONCEPTO} &
	    \multicolumn{1}{c|}{3} &
	    \multicolumn{1}{c|}{\%$\delta(3-4)$} &
	    \multicolumn{1}{c|}{4} &
	    \multicolumn{1}{c|}{\%$\delta(4-5)$} &
	    \multicolumn{1}{c}{5} \\
	%--------------------------------------------------
	\hline
	\hline
	1	&	INGRESOS NETOS                                       &  16,808,400.00 	&	14.37	&	 19,224,000.00 	&	12.36	&	 21,600,000.00  \\
	\hline
	\hline
	2	&	\multicolumn{6}{l}{COSTOS VARIABLES}                \\
	\hline
	3	&	Papelería                                            &  35,840.07 	&	36.69	&	 48,990.69 	&	17.01	&	 57,323.27  \\
	4	&	Material Didáctico                                   &  42,600.60 	&	36.96	&	 58,344.30 	&	19.00	&	 69,429.72  \\
	5	&	Material talleres                                    &  99,225.00 	&	40.00	&	 138,915.00 	&	31.25	&	 182,325.94  \\
	\hline
	6	&	TOTAL COSTOS VARIABLES                               &  177,665.67 	&	38.60	&	 246,249.99 	&	25.51	&	 309,078.93  \\
	\hline
	7	&	UTILIDAD BRUTA                                       &  16,630,734.33 	&	14.11	&	 18,977,750.01 	&	12.19	&	 21,290,921.07  \\
	\hline
	\hline
	8	&	\multicolumn{6}{l}{COSTOS FIJOS}                    \\
	\hline
	9	&	Servicios subcontratados                             &  1,649,340.00 	&	5.00	&	 1,731,807.00 	&	5.00	&	 1,818,397.35  \\
	10	&	Servicios (agua, luz, telefono)                      &  573,300.00 	&	5.00	&	 601,965.00 	&	5.00	&	 632,063.25  \\
	11	&	Sueldos Directos                                     &  5,915,755.70 	&	31.59	&	 7,784,415.02 	&	15.45	&	 8,986,902.29  \\
	\hline
	12	&	TOTAL COSTOS FIJOS                                   &  8,138,395.70 	&	24.33	&	 10,118,187.02 	&	13.04	&	 11,437,362.89  \\
	\hline
	13	&	MARGEN DE UTILIDAD                                   &  8,492,338.63 	&	4.32	&	 8,859,562.99 	&	11.22	&	 9,853,558.19  \\
	\hline
	\hline
	14	&	\multicolumn{6}{l}{GASTOS DE OPERACIÓN}             \\
	\hline
	15	&	\multicolumn{6}{l}{GASTOS DE ADMINISTRACIÓN}        \\
	\hline
	16	&	Sueldos Indirectos                                   &  2,284,075.56 	&	2.50	&	 2,341,177.45 	&	2.50	&	 2,399,706.89  \\
	17	&	Sueldos Administrativos                              &  2,569,585.01 	&	2.50	&	 2,633,824.63 	&	2.50	&	 2,699,670.25  \\
	18	&	Depreciación anual                                   &  610,804.65 	&	4.94	&	 640,980.69 	&	-23.40	&	 490,980.69  \\
	\hline
	19	&	\multicolumn{6}{l}{GASTOS DE VENTA}                 \\
	\hline
	20	&	Publicidad                                           &  198,450.00 	&	5.00	&	 208,372.50 	&	5.00	&	 218,791.13  \\
	\hline
	21	&	TOTAL DE GASTOS DE OPERACIÓN                         &  5,662,915.21 	&	2.85	&	 5,824,355.27 	&	-0.26	&	 5,809,148.94  \\
	\hline
	22	&	\multicolumn{6}{l}{GASTOS Y PRODUCTOS FINANCIEROS}  \\
	\hline
	23	&	Gastos Financieros                                   &  195,343.92 	&	-35.85	&	 125,314.48 	&	-62.97	&	 46,403.55  \\
	\hline
	24	&	TOTAL GASTOS FINANCIEROS                             &  195,343.92 	&	-35.85	&	 125,314.48 	&	-62.97	&	 46,403.55  \\
	\hline
	25	&	UTILIDAD DE OPERACIÓN                                &  2,634,079.50 	&	10.47	&	 2,909,893.25 	&	37.39	&	 3,998,005.69  \\
	\hline
	\hline
	26	&	\multicolumn{6}{l}{OTROS GASTOS Y PRODUCTOS}        \\
	\hline
	27	&	Amortización del Activo Diferido                     &  0.0 	&		&	 0.0 	&		&	 0.0  \\
	\hline
	28	&	UTILIDAD ANTES DE IMPUESTOS                          &  2,634,079.50 	&	10.47	&	 2,909,893.25 	&	37.39	&	 3,998,005.69  \\
	\hline
	\hline
	29	&	Provisión de ISR                                     &  737,542.26 	&	10.47	&	 814,770.11 	&	37.39	&	 1,119,441.59  \\
	30	&	Provisión de PTU                                     &  263,407.95 	&	10.47	&	 290,989.32 	&	37.39	&	 399,800.57  \\
	32	&	Impuesto al Depósido en Efectivo                     &  336,168.00 	&	14.37	&	 384,480.00 	&	12.36	&	 432,000.00  \\
	\hline
	33	&	TOTAL IMPUESTOS Y PRESTACIONES                       &  1,337,118.21 	&	11.45	&	 1,490,239.43 	&	30.93	&	 1,951,242.16  \\
	\hline
	\hline
	34	&	UTILIDAD DEL EJERCICIO                               &  1,296,961.29 	&	9.46	&	 1,419,653.81 	&	44.17	&	 2,046,763.53  \\
	%--------------------------------------------------
	\hline
	\multicolumn{7}{l}{\footnotesize Fuente: Elaboración Propia, 2010.}
    \end{tabular}
\end{table}


\clearpage
\section{Razones financieras}

Las razones financieras ofrecen un panorama revelador acerca de las principales fortalezas y áreas de oportunidad de una empresa\footnote{Para esta sección, se consultaron las siguientes fuentes: \citep{Brigham2005fundamentos, Gitman2003principios, Horngren2000introduccion, Moyer2004administracion, Van2003fundamentos}; y se elaboró una redacción propia.}.

Se dividen en tres categorías: de liquidez, apalancamiento y rentabilidad.

\begin{table}[h]
    \caption{Razones Financieras}
    \label{tbl:Razones}
    \centering
    \footnotesize
    \begin{tabular}{l*{5}{|c}}
        %--------------------------------------------------
        \multicolumn{1}{c|}{} &
            \multicolumn{5}{c}{AÑO} \\
        \cline{2-6}
        \multicolumn{1}{c|}{RAZON} &
            1 & 2 & 3 & 4 & 5 \\
        \hline
        \hline
	\multicolumn{6}{c}{RAZONES DE LIQUIDEZ} \\
	\hline
        Capital de Trabajo	&	\$514,467 	&	\$1,664,488 	&	\$3,299,104 	&	\$4,399,738 	&	\$6,937,482  \\
        Liquidez	&	1.22	&	1.89	&	3.49	&	7.28	&	N/A \\
        Prueba del ácido	&	0.98	&	1.28	&	2.17	&	3.86	&	N/A \\
        \hline
	\multicolumn{6}{c}{RAZONES DE APALANCAMIENTO} \\
	\hline
        Endeudamiento	&	23\%	&	18\%	&	12\%	&	6\%	&	0\% \\
        Apalancamiento Total	&	31\%	&	22\%	&	14\%	&	6\%	&	0\% \\
        Solvencia	&	77\%	&	82\%	&	88\%	&	94\%	&	100\% \\
        \hline
	\multicolumn{6}{c}{RAZONES DE RENTABILIDAD} \\
	\hline
        Rotación de la Inversión	&	0.56	&	1.09	&	1.53	&	1.63	&	1.64 \\
        Rotación de Activos Fijos	&	0.78	&	1.55	&	2.50	&	3.01	&	3.22 \\
        Utilidad Neta en Ventas	&	-26\%	&	6\%	&	8\%	&	7\%	&	9\% \\
        Control de Gastos	&	-2.14	&	5.24	&	4.37	&	4.10	&	2.84 \\
        Rentabilidad de Capital	&	-19\%	&	8\%	&	13\%	&	13\%	&	16\% \\
        Rendimientos de los Activos	&	-15\%	&	7\%	&	12\%	&	12\%	&	16\% \\
        %--------------------------------------------------
    \hline
    \multicolumn{6}{l}{\footnotesize Fuente: Elaboración Propia, 2010.}
    \end{tabular}
\end{table}



\subsection{Razones de Liquidez}

Miden la capacidad de la empresa para cubrir sus obligaciones a corto plazo. Se describen las siguientes:

\begin{description}
    \item[Capital de Trabajo         ] $ Activo Circulante-Pasivo Circulante                     $ \hfill \\
    Es el excedente de los activos de corto plazo sobre los pasivos de corto plazo, es una medida de la capacidad que tiene una empresa para continuar con el normal desarrollo de sus actividades en el corto plazo. Una cantidad negativa indicaría la necesidad de endeudamiento de corto plazo. En este proyecto se observa que en todos los años es mayor a cero.
    \item[Liquidez                   ] $ \frac{Activo Circulante}{Pasivo Circulante}             $ \hfill \\
    Muestra la capacidad de la empresa para responder a sus obligaciones de corto plazo con sus activos circulantes. Un índice menor a uno indica incapacidad para cubrir las deudas de corto plazo con los activos de corto plazo. El bachillerato en todos los años es capaz de afrontar sus compromisos a corto plazo con el activo circulante.
    \item[Prueba del ácido           ] $ \frac{Activo Circulante-Activo Menos líquido}{Pasivo Circulante} $ \hfill \\
    Similar a la razón de liquidez, indica la capacidad de responder a compromisos de corto plazo con los activos de corto plazo. La diferencia es que se hace énfasis en que se trate de los activos más líquidos, pues con frecuencia, algunas cuentas de activo circulante encuentran dificultad para convertirse en efectivo, por ejemplo, los inventarios. Aplican los mismos criterios que en la razón de liquidez. A excepción del primer año, este indicador es siempre superior a 1. Cabe hacer notar, que por motivos de simplicidad, no se desglosó el pasivo en circulante y fijo, sino que se está utilizando el activo total para ambas pruebas; esto implica una prueba mucho más estricta: cubrir con los activos circulantes la deuda de largo plazo.\footnote{El numerador original se expresa como $Activo Circulante-Inventarios$, sin embargo, el concepto más amplio es quitar del activo circulante los activos menos líquidos.}
\end{description}

\subsection{Razones de Apalancamiento}

Permiten analizar la estructura de la deuda de una empresa y su capacidad para asumir compromisos de largo plazo. Se describen las siguientes:

\begin{description}
    \item[Endeudamiento              ] $ \frac{Pasivo Total}{Activo Total}                       $ \hfill \\
    Determina el porcentaje en que se están financiando los activos con deuda. Un porcentaje elevado implica vulnerabilidad de la empresa ante un revés en el negocio o la necesidad de pagar a los acreedores. Un porcentaje excesivamente bajo o nulo significa una pérdida de oportunidad de reinversión. Este proyecto comienza con un endeudamiento de 23\% y cierra sin deudas; lo cual lo capacita para adquirir nuevamente deuda. En principio no se tiene previsto para el segundo lustro un endeudamiento similar al del primero aunque está abierta la posibilidad de adquirir deudas menores, principalmente para renovar equipo y realizar remodelaciones menores.
    \item[Apalancamiento Total       ] $ \frac{Pasivo Total}{Capital Total}                      $ \hfill \\
    Similar a la razón de endeudamiento pero analizando la relación entre el pasivo y el capital para formar el activo. El bachillerato tecnológico inicia operaciones con un apalancamiento de 31\%, mismo que descenderá hasta desaparecer al quinto año en que se haya pagado completamente la deuda.
    \item[Solvencia                  ] $ \frac{Capital Total}{Activo Total}                      $ \hfill \\
    Es la parte proporcional con la que contribuye el capital para formar el activo. Indica el grado de independencia de la institución respecto a otras entidades acreedoras. En términos generales, la escuela es solvente.
\end{description}

\subsection{Razones de Rentabilidad}

Se utilizan para medir la capacidad de la empresa para generar utilidades con base en los recursos de que dispone. Se describen las siguientes:

\begin{description}
    \item[Rotación de la Inversión   ] $ \frac{Ventas}{Activos Totales}                          $ \hfill \\
    Indica la eficiencia con que la empresa puede utilizar sus activos para generar ventas. Valores menores a uno implican pérdida para el inversionista. Mientras mayor es el número mayor la eficiencia en la utilización de los recursos. La escuela inicia siendo poco eficiente (0.56) y hacia el final del periodo tiende a estabilizarse en 1.64.
    \item[Rotación de Activos Fijos  ] $ \frac{Ventas}{Activos Fijos}                            $ \hfill \\
    Mide la capacidad de los activos fijos para generar ventas. Aplica el mismo criterio que en la rotación de la inversión. El bachillerato inicia con 0.78 y termina en 3.22. Aparentemente, a medida que pasa el tiempo, el uso del activo fijo se vuelve más eficiente. Sin embargo hay que considerar el impacto de la depreciación. Es posible que simplemente se trate de una disminución del valor del activo fijo más que una mejora en su utilización. Aun así, atendiendo el comportamiento de la rotación de la inversión, se puede afirmar que al menos en parte, se mejora la eficiencia del uso del activo fijo.
    \item[Utilidad Neta en Ventas    ] $ \frac{Utilidad neta}{Ventas}                            $ \hfill \\
    Determina el porcentaje que representa la utilidad neta respecto de las ventas. Mientras mayor es el número, la eficiencia para obtener recursos en la operación es mayor. El proyecto inicia con pérdidas de 26\% debido a la reducida matrícula estudiantil al comienzo; al final se estabiliza en 9\%.
    \item[Control de Gastos          ] $ \frac{Gastos de Operación}{Utilidad Neta}               $ \hfill \\
    Indica la eficiencia con la que se opera una empresa. Un número muy alto significa que la administración es muy pesada y probablemente haya que considerar un esquema más austero o ágil. Un número negativo simplemente indica pérdidas. El proyecto inicia con pérdidas el primer año, en el segundo año, el gasto de operación es cinco veces superior a las utilidades. Al final del periodo esta cantidad se reduce a 2.84. En principio este número puede parecer elevado, sin embargo corresponde con el comportamiento de la industria en general, en el sentido de que la nómina tanto docente como administrativa constituye el rubro más pesado en una institución educativa.
    \item[Rentabilidad de Capital    ] $ \frac{Utilidad Neta}{Capital}                           $ \hfill \\
    Mide la capacidad de creación de valor a partir del capital invertido. Cantidades menores a uno implican pérdida de los socios mientras que cantidades superiores implican ganancia. La escuela inicia con pérdidas (19\%) y cierra con una ganancia de 16\%.
    \item[Rendimientos de los Activos] $ \frac{Utilidad Neta}{Activo Total}                      $ \hfill \\
    Permite calcular la utilidad generada por el uso del activo del negocio. Una cantidad pequeña indica una pobre administación del activo para producir utilidades, mientras que una cantidad alta significa una buena administración del activo. El bachillerato tecnológico inicia con -15\% y termina con 16\%.
\end{description}

\subsection{Observaciones Generales}

Para el caso del presente proyecto se puede ver una solidez en cuanto a la liquidez (razones de liquidez) y capacidad de asumir compromisos (deuda) en el corto y el largo plazo (razones de apalancamiento). Por otra parte, las razones de rentabilidad indican que se trata de un proyecto que genera ganancias aunque a un nivel modesto que pudiera intentar incrementarse. Sin embargo, dado que es un proyecto de interés social, no se busca la maximización de una utilidad sino hacer más accesible la educación de calidad a la población de escasos recursos. Se amplía este tema en las conclusiones finales del presente trabajo.

