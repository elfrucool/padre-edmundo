\chapter{Don Bosco - Sistema Preventivo - Nota 1}
\label{ap:preventivo}
San Juan Bosco, en \emph{El Sistema Preventivo en la Educación de la Juventud} (1877) en la nota 1, dice textualmente: <<No hace mucho tiempo que un ministro de la reina de Inglaterra, visitando un colegio de Turín, fue conducido a una amplia sala donde estudiaban unos quinientos jóvenes. Fue grande su maravilla cuando observó tan gran multitud de chicos en perfecto silencio y sin asistentes. Se maravilló aún más al saber que a lo largo del año no se había registrado ninguna palabra que distrajera, ningún motivo para infligir ni amenazar ningún castigo. `---¿Cómo es posible obtener tanto silencio y tanta disciplina?', preguntó, `dígamelo'. Y vos --- añadió al secretario --- tomad nota de cuanto se diga'. --- `Señor', respondió el director del centro, `el medio que usamos nosotros, no pueden usarlo ustedes'. --- `¿Por qué?' --- `Son arcanos revelados solamente a los católicos'. --- `¿Cuáles son?' --- `La frecuente confesión y comunión y la misa diaria bien oída' --- `Tiene usted razón, nos faltan estos medios de educación. ¿No pueden ser suplidos por otros?' --- Si no se usan estos recursos religiosos, hay que recurrir a las amenazas y al palo'. --- `Tiene usted razón, tiene usted razón. \emph{O religión o palo}, lo contaré en Londres'.>> \citep{Canals95}
