\chapter{Administración de Negocios}
\label{ch:AdministracionNegocios}

Este capítulo analiza las características sociodemográficas de la región donde se llevará a cabo el proyecto para determinar las condiciones del mercado.

Comienza con las \emph{\nameref{sec:Neg:NecesidadesMercado}}, donde se resaltan aspectos sociodemográficos como el rezago educativo y los problemas sociales que enfrenta la localidad. Posteriormente se traza el perfil de los \emph{\nameref{sec:Clientes}} a quienes se les ofrecerá el servicio y de los \emph{\nameref{sec:Neg:Competidores}} frente a quienes este proyecto deberá diferenciarse.

Con estos datos se puede establecer una estrategia, la cual es descrita en el \emph{\nameref{sec:Neg:Foda}} y servirá como auxiliar en la elaboración de la estrategia de mercado descrita en el capítulo \ref{ch:AdministracionMercados} <<\emph{\nameref{ch:AdministracionMercados}}>>.

Finalmente en las \emph{\nameref{sec:Neg:Estimaciones}} se determina el tamaño y valor del mercado así como la matrícula esperada. Esta información constituye el insumo más importante para el capítulo \ref{ch:AdministracionOperaciones} <<\emph{\nameref{ch:AdministracionOperaciones}}>> donde se establecen los presupuestos de ingresos y egresos del proyecto.

\section{Necesidades del Mercado}
\label{sec:Neg:NecesidadesMercado}

Las necesidades del mercado se pueden revisar desde tres aspectos: rezago educativo, problemas sociales y necesidades de otros actores involucrados (empresas y gobierno). Posteriormente se revisan algunas tendencias y al final se incluye un apartado sobre nuestra propuesta de valor considerando esta problemática.

\subsection{Rezago Educativo}

Como se señaló en la subsección \ref{sub:Intro:AspectosPrevios} (página \pageref{sub:Intro:AspectosPrevios}), Guanajuato es una de las entidades con mayor rezago educativo del país. Paradójicamente existe un gran desarrollo industrial que abarca desde León hasta Salamanca.\footnote{Por ejemplo, en Silao se encuentra una de las plantas de General Motors}

Según el conteo de población y vivienda de 2005 \citep{Inegi2005}, la población total entre 15 y 19 años en el municipio de Irapuato, fue de 47,177 jóvenes, de los cuales 22,117 asistían a la escuela, lo cual representa un 46\%. Más aun, la misma fuente informa que de 16,086 jóvenes comprendidos en esas edades con secundaria terminada, únicamente 14,623 asisten al bachillerato (ver el cuadro \ref{tbl:INEGI:PoblacionEstudiaIrapuato}); hecho confirmado por la Secretaría de Educación de Guanajuato (SEG), según la cual, la matrícula de bachillerato en 2005 fue de 15,687 estudiantes y en 2009 de 17,734 (\citep{Seg2010}).

\begin{table}
	\centering
    \caption{Datos Generales de Poblaci\'on}
    \label{tbl:INEGI:PoblacionEstudiaIrapuato}
    \begin{tabular}{p{4in}|r|r}
        \multicolumn{1}{c|}{VARIABLE}
        	& \multicolumn{1}{c|}{2000$^{/a}$}
        	& \multicolumn{1}{c}{2005$^{/b}$} \\
        \hline \hline
        Poblaci\'on total & 440,134 & 463,103 \\
        Poblaci\'on de 15 a 19 a\~nos & 45,325 & 47,177 \\
        Poblaci\'on de 15 a 19 a\~nos que asiste a la escuela & 17,607 & 22,117 \\
        Poblaci\'on de 15 a 19 a\~nos con secundaria terminada & 12,688 & 16,086 \\
        Poblaci\'on de 15 a 19 a\~nos con estudios de bachillerato$^{/c}$ & 10,109 & 14,623 \\
        \hline
        \multicolumn{3}{l}{$^{/a}$ \footnotesize Fuente: \citep{Inegi2000}} \\
        \multicolumn{3}{l}{$^{/b}$ \footnotesize Fuente: \citep{Inegi2005}} \\
        \multicolumn{3}{p{5.4in}}{$^{/c}$ \footnotesize No especifica si est\'a cursando o ya curs\'o, por el desglose de cada a\~no dentro del grupo de edad de 15 a 19 a\~nos se intuye que se refiere a ``cursando''}
    \end{tabular}
\end{table}


Entre 2005 y 2009, se promedió anualmente un aproximado de 830 egresados de secundaria que no ingresaron a bachillerato. En el mismo periodo, el índice de aprobación en bachillerato fue de 58\%; el número de egresados de secundaria se incrementó anualmente 2.3\% en promedio; el de nuevos ingresos a bachillerato fue de 1.82\% y el de egresos fue de 1.75\% \citep{Seg2010}\footnote{Ver los cuadros \ref{tbl:SEG:Secundaria} y \ref{tbl:SEG:Bachillerato}}.

¿Por qué tan pocos jóvenes estudian? Una primera explicación es la común para estos casos: \emph{Los jóvenes de escasos recursos abandonan la escuela porque tienen que trabajar para sostener a su familia; encuentran dificultades para emplearse dignamente pues carecen de los conocimientos necesarios. Para muchos, estudiar una carrera universitaria se vuelve un privilegio.}

El Consejo Nacional de Evaluación de la Política de Desarrollo Social (CONEVAL) informa que Irapuato es el quinto municipio con menor índice de pobreza del estado, con un 47\% de población en pobreza patrimonial, 21\% en pobreza de capacidades y 14\% en pobreza alimentaria \citep{Coneval2009} (cuadro \ref{tbl:CONEVAL:MunicipiosPobreza}).

Una segunda explicación proviene de la falta de espacios educativos; en efecto, según la Secretaría de Educación de Guanajuato \citep{Seg2010}, mientras  Irapuato mantiene desde el ciclo esclolar 2004-2005 a la fecha, un índice de absorción escolar promedio de 88\%\footnote{\emph{Absorción}: proporción de alumnos de nuevo ingreso a primer grado de un nuvel educativo, respecto a los alumnos egresados del nivel y ciclo inmediato anterior \citep{Seg2010}.};
municipios como Tierra Blanca o Celaya, mantienen un promedio superior a 110\% en el mismo periodo.

\begin{table}
    \centering
    \caption{Egresados de Secundaria en el Municipio de Irapuato}
    \label{tbl:SEG:Secundaria}
    \begin{tabular}{c|c||l|r}
        AÑO & EGRESADOS & \multicolumn{1}{c|}{INDICADOR}
            & \multicolumn{1}{c}{VALOR} \\
        \hline \hline
        2005 & 7,515 & Acumulado       & 38,560  \\
        2006 & 7,561 & Promedio        &  7,712  \\
        2007 & 7,306 & Incremento      &  177.4  \\
        2008 & 7,991 & Incremento \%   &  2.30\% \\
        2009 & 8,187 & Proyectado 2010 &  8,244  \\
        \hline
        \multicolumn{4}{l}{\footnotesize Fuente: \citep{Seg2010}}
    \end{tabular}
\end{table}


\begin{table}
    \centering
    \caption{Indigadores Generales del Bachillerato en Irapuato}
    \footnotesize
    \label{tbl:SEG:Bachillerato}
    \begin{tabular}{c|r|r|r|r}
        AÑO  & NUEVO INGRESO & EGRESADOS & MATRICULA & APROBACION \\
        \hline \hline
        2005 & 6,716 & 3,856 & 15,687 & 57\% \\
        2006 & 7,112 & 3,882 & 16,242 & 55\% \\
        2007 & 6,353 & 3,797 & 15,457 & 60\% \\
        2008 & 6,669 & 4,228 & 15,682 & 63\% \\
        2009 & 7,564 & 3,969 & 17,374 & 52\% \\
        \hline
        \multicolumn{5}{c}{INDICADORES} \\
        \hline
        \multicolumn{1}{l|}{Acumulado}       & 34,414  & 19,732  & 80,442 &     N/A \\
        \multicolumn{1}{l|}{Promedio}        &  6,883  &  3,946  & 16,088 &    58\% \\
        \multicolumn{1}{l|}{Crecimiento}     &  125.3  &  57.2   &  281.4 & -0.11\% \\
        \multicolumn{1}{l|}{Crecimiento \%}  &  1.82\% &  1.45\% & 1.75\% & -0.19\% \\
        \multicolumn{1}{l|}{Proyectado 2010} &  7,259  &  4,118  & 16,933 &    57\% \\
        \hline
        \multicolumn{5}{l}{\footnotesize Fuente: \citep{Seg2010}}
    \end{tabular}
\end{table}

\begin{table}
    \centering
    \caption[Municipios con Mayor y Menor Grado de Pobreza por Ingresos]{Municipios con Mayor y Menor Grado de Pobreza\newline por Ingresos}
    \label{tbl:CONEVAL:MunicipiosPobreza}
    \footnotesize
    \begin{tabular}{c|l|r||c|c|c}
                       &                &           & POBREZA     & POBREZA DE  & POBREZA DE \\
                       & MUNICIPIO      & POBLACIÓN & ALIMENTARIA & CAPACIDADES & PATRIMONIO \\ 
        \hline \hline
        Estatal        & Guanajuato     & 4,893,812 & 18.9 \%     & 26.6 \%     & 51.6 \%    \\ 
        \hline
        Municipios     & Xichú          &    10,592 & 60.8 \%     & 68.5 \%     & 83.6 \%    \\ 
        con mayor      & Atarjea        &     5,035 & 58.8 \%     & 67.1 \%     & 83.2 \%    \\ 
        porcentaje     & Tierra Blanca  &    16,136 & 50.9 \%     & 59.4 \%     & 77.3 \%    \\ 
        de pobreza     & Victoria       &    19,112 & 47.4 \%     & 55.4 \%     & 73.2 \%    \\ 
                       & Santa Catarina &     4,544 & 41.3 \%     & 47.8 \%     & 64.4 \%    \\ 
        \hline
        Municipios     & Irapuato       &   463,103 & 14.0 \%     & 21.2 \%     & 46.9 \%    \\ 
        con menor      & Uriangato      &    53,077 & 11.6 \%     & 19.9 \%     & 50.4 \%    \\ 
        porcentaje     & Celaya         &   415,869 & 11.5 \%     & 17.9 \%     & 41.3 \%    \\ 
        de pobreza     & Moroleón       &    46,751 &  9.5 \%     & 16.1 \%     & 42.6 \%    \\ 
                       & León           & 1,278,087 &  7.9 \%     & 13.6 \%     & 38.2 \%    \\ 
        \hline
        \multicolumn{6}{l}{\footnotesize Fuente: \citep{Coneval2009}}
    \end{tabular}
\end{table}


Otro indicador importante es la atención a la demanda estatal \footnote{\emph{Atención a la demanda estatal}: Valores cercanos a 100 implican que el sector educativo atiende a la mayoría de la población en edad escolar que demande este servicio \citep{Seg2010}}, Irapuato mantiene del ciclo escolar 2004-2005 al ciclo 2009-2010 un promedio de 81\%, municipios como Uriangato o Tierra Blanca tienen más del 90\%. En el último ciclo escolar, el municipio de Atarjea obtuvo 120\% en este indicador.

Según estas cifras, a nivel bachillerato, el municipio no ha cubierto el total de la demanda educativa, ante este escenario, los jóvenes buscan alternativas en otro sitio o simplemente abandonan la escuela.

Finalmente está la valoración de la educación; muchas personas consideran que con saber leer y escribir y hacer cuentas basta; este fenómeno se da particularmente entre quienes se dedican al comercio y elaboración de productos artesanales. Para estas personas, la educación superior es un saber que no aporta valor económico, muy teórico y alejado de sus necesidades.\footnote{Convendría realizar un estudio de mercado más afondo sobre este tema, dicho estudio rebasa el alcance de este proyecto.}

\begin{table}
	\centering
    \caption{\'Indices de Absorci\'on y de Atenci\'on a la Demanda Estatal}
    \label{tbl:SEG:AbsorcionYDemandaEstatal}
    \footnotesize
    \begin{tabular}{l||r|r|r|r|r|r||r}
        \hline
        \hline
                      & \multicolumn{7}{|c}{ABSORCION}                                                  \\
        \hline
        MUNICIPIO     & 04-05     & 05-06     & 06-07     & 07-08     & 08-09     & 09-10     & PROMEDIO \\
        \hline
        Estado        &  80.40\%  &  83.00\%  &  87.80\%  &  84.90\%  &  86.00\%  &  86.40\%  &  84.75\% \\
        Tierra Blanca & 125.00\%  & 130.90\%  & 133.30\%  & 115.40\%  & 119.10\%  & 120.50\%  & 124.03\% \\
        Celaya        & 106.50\%  & 108.60\%  & 121.30\%  & 113.60\%  & 120.00\%  & 117.60\%  & 114.60\% \\
        Irapuato      &  88.60\%  &  89.40\%  &  94.10\%  &  87.00\%  &  83.50\%  &  85.70\%  &  88.05\% \\
        \hline
        \hline
                      & \multicolumn{7}{|c}{ATENCION A LA DEMANDA ESTATAL}                    \\
        \hline
        MUNICIPIO     & 04-05   & 05-06   & 06-07   & 07-08   & 08-09    & 09-10    & PROMEDIO \\
        \hline
        Estado        & 76.90\% & 77.90\% & 82.20\% & 79.00\% &  81.10\% &  81.00\% & 79.68\%  \\
        Uriangato     & 84.20\% & 90.10\% & 94.30\% & 87.50\% & 147.30\% &  88.20\% & 98.60\%  \\
        Tierra Blanca & 89.00\% & 95.40\% & 90.20\% & 86.40\% &  97.40\% &  88.50\% & 91.15\%  \\
        Atarjea       &  0.00\% &  0.00\% &  0.00\% &  0.00\% &  20.00\% & 119.80\% & 23.30\%  \\
        Irapuato      & 81.20\% & 81.30\% & 83.90\% & 78.30\% &  81.60\% &  80.50\% & 81.13\%  \\
        \hline
        \multicolumn{8}{l}{Fuente: \citep{Seg2010}}
    \end{tabular}
\end{table}


\subsection{Problemas Sociales}

Se mencionan dos problemas sociales derivados de una inadecuada educación: la ruptura familiar y la desigualdad de oportunidades por relaciones humanas.

La ruptura familiar provoca desorientación y violencia en la juventud; existen muchos j\'ovenes que caen en las drogas con todos los problemas que ello implica\footnote{Entrevista del autor con el Pbro. Edmundo Morales, S.D.B., 16 de octubre de 2010}; la causa es la falta de valores: la falta de orientación y valores en la juventud los incapacita para afrontar la vida con integridad, los vuelve vulnerables a diversas formas de manipulación y violencia. Es significativo el elevado índice de divorcios en la entidad, síntoma de una falta de capacidad de establecer relaciones estables y compromisos duraderos\footnote{Cf. \citep{Morales09}}.

En un diagnóstico reciente, la comunidad salesiana local identificó la drogadicción como un problema entre muchos jóvenes, así como el aumento de pandillas y violencia en las calles. Consideran que la educación de esos jóvenes es el camino para resolver esos problemas\footnote{Diagnóstico interno de la comunidad, 2010}.

Otro problema muy grave es la desigualdad en cuanto a oportunidades se refiere\footnote{Aquí no se hace referencia a los índices de desigualdad social sino a un aspecto cualitativo del problema: las relaciones humanas}. La división entre los que van a una escuela de paga y los que van a escuela de gobierno, entre la gente <<bien>> y el <<pueblo>>, provoca un serio problema de discriminación alimentado por prejuicios que al final lo que produce es que unos cuantos se beneficien del conocimiento y las relaciones personales para obtener empleo y el resto tenga serias dificultades a pesar de que en muchos casos existe talento y compromiso.

\subsection{Otros Actores Involucrados}

Las empresas necesitan personal capacitado y competente, la educación formal es insuficiente, pues no ha mostrado capacidad de adaptarse al dinamismo del mundo laboral.

Finalmente, los gobiernos de los tres órdenes (federal, estatal y municipal) enfrentan un grave reto ante los elevados índices de analfabetismo que entran en conflicto con el creciente desarrollo industrial, el riesgo es que las empresas contraten <<personal de fuera>> y los habitantes de la región se vuelvan <<meros expectadores del progreso>>.

\subsection{Tendencias identificadas}

Se han identificado las siguientes tendencias:

\begin{itemize}
	\item En cuanto al rezago educativo, tanto los gobiernos como la sociedad han realizado notables esfuerzos por superar el rezago educativo. Aun así, todavía resulta insuficiente.
	\item Por otra parte, las empresas requieren de manera creciente los servicios de personal mejor capacitado, esto para afrontar un ambiente global cada vez más competitivo. \emph{La verdadera preparación para el trabajo comienza en el trabajo mismo}.
	\item La crisis de valores se acentúa en proporción directa con la desintegración familiar, el incremento de la violencia y el desmoronamiento del tejido social. Es cada vez mayor el número de jóvenes y adultos que carecen de un criterio orientador de su vida, lo cual los vuelve incapaces de asumir compromisos duraderos y frágiles en el momento de enfrentar la tentación de elegir el camino más fácil en lugar del camino correcto.
	\item Existe una mayor demanda de espacios educativos la cual ha sido atendida con insuficiencia por parte de los gobiernos y otros actores de la sociedad.
\end{itemize}

\subsection{Propuesta de Valor}

Para los jóvenes de la región un bachillerato de calidad y una carrera técnica; para los de escasos recursos, ofrecemos un precio accesible con posibilidad de becas y financiamiento que les permita encontrar un empleo y continuar con una carrera universitaria.

Una formación en valores que les permitan afrontar con entereza las dificultades de la vida, convirtiéndose en \emph{buenos cristianos y honestos ciudadanos}.\footnote{Lema escogido por Don Bosco para sus escuelas}

Trato igual y convivencia entre los que pagan colegiatura y los que no, una convivencia que supere los prejuicios entre clases sociales y elimine las barreras de discriminación.

Para las empresas contar con talento humano adecuado a sus necesidades adaptando el plan de estudios y ofreciendo carreras técnicas adecuadas a las necesidades laborales más apremiantes.

Para los gobiernos reducir los índices de analfabetismo, rezago educativo y desempleo: y a largo plazo, los índices de violencia y ruptura familiar.

\section{Clientes}
\label{sec:Clientes}

\subsection{Pirámide de Necesidades del Mercado}

En la base se encuentra el grupo más numeroso, constituido por jóvenes entre 15 y 19 años de edad, con secundaria terminada y deseosos de continuar sus estudios.

Inmediatamente después están los padres de familia, preocupados por la educación de sus hijos y que reciban valores.

Un grupo importante es el de las empresas, interesadas en disponer de mano de obra calificada.

En la cúspide se encuentran los gobiernos de los tres órdenes (federal, estatal y municipal), quienes tienen interés en reducir los índices de analfabetismo y rezago educativo.

El proyecto se enfoca principalmente a los dos primeros grupos (jóvenes y padres de familia), de quienes se presentan sus perfiles a continuación.

\subsection{Grupos de clientes}

\subsubsection{Jóvenes}

En Irapuato, los j\'ovenes entre 15 y 19 a\~nos de edad con secundaria terminada conforman un grupo de aproximadamente 16,086 integrantes \citep{Inegi2005}. La mayor\'{i}a sigue estudiando el bachillerato. Entre sus gustos e intereses est\'a el deporte; especialmente el futbol. La afici\'on del Irapuato es fuerte en el municipio\footnote{Entrevista del autor con el Pbro. Edmundo Morales, S.D.B., 16 de octubre de 2010}. Otros intereses de los j\'ovenes son el cine, la m\'usica y las fiestas \citep{Cordero07}.

Entre los intereses de los jóvenes está el estudio, pues muchos cambian de ciudad en busca de espacios educativos\footnote{Entrevista del autor con el Pbro. Edmundo Morales, S.D.B., 16 de octubre de 2010}.

Para efectos de la estrategia a seguir, se utilizarán los parámetros como unidad de segmentación.

La escuela busca cubrir un amplio espectro de jóvenes, los cuales se pueden ubicar en un espectro que va desde los que tienen una posición económica acomodada que les permite pagar una costosa educación, hasta los que abandonan la escuela por falta de medios económicos.

%Algunos grupos de estudiantes, de acuerdo a su criterio de elección escolar, son:
%\begin{itemize}
%	\item
%	Quienes buscan un colegio particular de prestigio
%	\item
%	Quienes buscan un colegio particular a precios más accesibles que el promedio
%	\item
%	Quienes no encontraron cupo en el sistema oficial y no tienen medios económicos para inscribirse a un colegio particular
%	\item
%	Quienes desean prepararse para el trabajo, sea porque desean pagar sus estudios universitarios trabajando o bien porque necesitan encontrar un buen trabajo para sostener a su familia.
%\end{itemize}

\subsubsection{Padres de familia}

Los padres de familia están interesados en procurar un mejor futuro a sus hijos. Forma parte de la mentalidad local que el camino a un mejor progreso es estudiar una carrera universitaria, en ese sentido, las carreras exclusivamente técnicas no son apreciadas\footnote{Entrevista del autor con el Pbro. Edmundo Morales, S.D.B., 16 de octubre de 2010}.

Uno de los problemas que enfrentan los padres de familia es el costo de las colegiaturas. Según la Encuesta de Ingresos y Gastos de los Hogares 2008 Guanajuato\footnote{\citep{INEGI-2009-DGES-003}}, el gasto promedio en educaci\'on por miembro de la familia es de \$ 235.20, siendo el m\'as bajo de \$ 0.00 y el m\'as alto de \$ 7,900.00.

Considerando que el bachillerato oficial cuesta \$ 3,000.00 al semestre (m\'as de \$ 500.00 mensuales) y los colegios particulares oscilan alrededor de los \$ 3,000.00 mensuales; resulta una verdadera preocupaci\'on para los padres de familia el cubrir los gastos de la educaci\'on de sus hijos.

Otro aspecto a considerar es la educacin en valores; Guanajuato tiene una población 96\% católica cuya característica es apreciar los valores de su fe\footnote{Inegi2000}. Es una sociedad, en este aspecto, tradicional\footnote{Entrevista del autor con el Pbro. Edmundo Morales, S.D.B., 16 de octubre de 2010}, uno de los centros religiosos más importantes del país se encuentra a pocos kilómetros de Irapuato\footnote{El santuario de Cristo Rey, ubicado en la cima del Cerro del Cubilete en Silao, Gto.; son emblemáticas las peregrinaciones juveniles que se realizan en enero cada año con repercusión en la prensa nacional y local}, ese centro recibe miles de peregrinos durante todo el año, tanto locales como de otras entidades; siendo también un importante centro turístico.

%En conclusión, se enlistan algunos grupos de padres de familia:
%
%\begin{itemize}
%	\item
%	Aquellos que buscan una educación de calidad para sus hijos
%	\item
%	Aquellos que desean que sus hijos reciban una preparación para el trabajo además del bachillerato
%	\item
%	Aquellos interesados en que sus hijos reciban valores y se formen como buenos cristianos
%	\item
%	Aquellos que no pueden pagar un bachillerato a sus hijos y desean que sigan estudiando
%\end{itemize}

\subsection{Perfilamiento}

Se identifican los siguientes parámetros de perfilamiento, tanto para estudiantes como para padres de familia, los mismos servirán como perfiles de clientes para efectos del presente trabajo:

\begin{description}
	\item [Posición económica] la decisión se basa en la economía
	\item [Interés académico] la decisión se basa en la calidad educativa, prestigio o calidad técnica
	\item [Interés laboral] la decisión se basa en ser un medio para obtener empleo
	\item [Orientación religiosa] se elige una escuela que vaya acorde con sus convicciones religiosas y valores
	\item [Otros intereses] ambiente, valores, cultura, actividades, facilidad, etc
\end{description}

Mediante estos parámetros se pueden identificar varios grupos de estudiantes:

\begin{itemize}
	\item Quienes buscan un colegio particular dentro de las posibilidades de pago (posición económica)
	\item Quienes buscan un colegio oficial porque uno particular está fuera de su alcance (posición económica)
	\item Quienes no alcanzaron cupo en el sistema oficial (posición económica)
	\item Quienes buscan ingresar al sistema oficial por su nivel académico (interés académico)
	\item Quienes buscan una preparación técnica sólida (interés académico)
	\item Quienes buscan un colegio particular de prestigio (interés académico)
	\item Quienes demandan capacitación para el empleo y de ser posible, continuar sus estudios (interés laboral)
	\item Quienes buscan un colegio con <<ambiente>> y actividades juveniles interesantes (otros intereses)
	\item Quienes buscan un colegio con orientación humanista (otros intereses: cultura)
	\item Quienes dejaron la escuela por motivos económicos y desean retomar sus estudios (combinado: económico, académico y laboral)
\end{itemize}

Y lo mismo para padres de familia, además de las decisiones económicas listadas arriba:

\begin{itemize}
	\item Los que buscan continuar con una tradición. Esta puede ser del sistema oficial o el particular. Que estudie en la misma escuela/sistema que su padre o su madre (otros intereses: cultura)
	\item Competitividad y relaciones. Los papás buscan lo mejor para sus hijos: el más alto nivel acadmico que puedan pagar y el ambiente social con las mejores relaciones posibles a futuro (interés académico y laboral).
	\item Educación católioca de fondo, valores humanos. La escuela que mejor les brinde esa posibilidad.
	\item Eligen una carrera corta que le permita a sus hijos incorporarse lo más rápido posible al mercado laboral y si pueden que se sostengan sus estudios posteriores. El bachillerato tecnológico es una buena alternativa. (interés laboral y posición económica)
\end{itemize}

\section{Competidores}
\label{sec:Neg:Competidores}

El municipio cuenta con 65 centros educativos de nivel medio superior, de los cuales 22 son oficiales y el resto particulares; sin embargo, el 60\% de la matrícula estudia en un bachillerato público.

Las escuelas públicas más importantes por su matrícula son la Escuela Preparatoria de Irapuato Universidad, el Centro de Bachillerato Tecnol\'ogico Industrial y el Colegio de Estudios Cient\'ificos y Tecnol\'ogico;
 en conjunto atienden a 3,877 estudiantes. Por su parte los colegios particulares con mayor número de estudiantes son el Centro de Estudios de Celaya, el Instituto Tecnol\'ogico de Superaci\'on Integral, el I.T.E.S.M. Campus Irapuato y el Colegio Pedro Mart\'inez V\'azquez (Maristas) cuya matrícula sumada asciende a 2,034 alumnos.

De estas escuelas, por su importancia y prestigio se seleccionaron las tres siguientes: Escuela Preparatoria de Irapuato Universidad, I.T.E.S.M. Campus Irapuato y Colegio Pedro Mart\'inez V\'azquez. En opini\'on del Pbro. Edmundo Morales, S.D.B., la mayoría de los jóvenes tiene interés estas escuelas y en ese orden\footnote{Entrevista con el autor, 16 de octubre de 2010}.

Se incluye además la preparatoria de la Universidad Qetzalcóatl debido a que ofrece carreras técnicas y colegiaturas accesibles\footnote{\texttt{http://www.uqi.edu.mx/informes/bachilleratos.aspx}}.

Las características m\'as importantes de estos centros educativos se muestran en el cuadro \ref{tbl:CompetidoresDetalle} y pueden resumirse, desde el punto de vista del cliente, de la siguiente forma: \emph{En Irapuato, el CPMV --como todas las instituciones maristas-- tiene un alto perfil humanista del que carecen muchas otras orientadas a la tecnología (ITESM). Sin embargo... las dos escuelas de enseñanza media superior mejor calificadas en Irapuato son la Prepa Oficial y en segundo lugar el Tec.}\footnote{Yahoo respuestas,\\
\texttt{http://mx.answers.yahoo.com/question/index; \_ylt=AnGhrxpJkWIr4VBc2yovwpHB8gt.; \_ylv=3?qid=20090622145127AADiunV}}

\begin{table}[t]
	\centering
    \caption{Matr\'icula Estudiantil en Bachilleratos P\'ublicos y Privados}
    \label{tbl:SEG:MatriculaPublicoPrivado}
    \begin{tabular}{l||c|r||r|r}
                   & \multicolumn{2}{c||}{ESCUELAS}
                        & \multicolumn{2}{c}{MATRICULA}    \\
        \hline
        TIPO       & NUMERO & \%       & NUMERO & \%       \\
        \hline
        \hline
        Oficial    & 22     &  33.85\% & 10,033 &  57.75\% \\
        Particular & 43     &  66.15\% &  7,341 &  42.25\% \\
        TOTAL      & 65     & 100.00\% & 17,374 & 100.00\% \\
        \hline
        \multicolumn{5}{l}{Fuente: \citep{Seg2010}}
    \end{tabular}
\end{table}


\begin{table}[t]
    \centering
    \caption{Participaci\'on del Mercado de Competidores Elegidos}
    \label{tbl:SEG:CompetidoresParticipacion}
    \begin{tabular}{l|r|r}
        INSTITUCION                                   & MATRICULA & \% DEL TOTAL \\
        \hline
        \hline
        Escuela Preparatoria de Irapuato Universidad  & 1,749     & 10.07\%      \\
        I.T.E.S.M. Campus Irapuato                    &   372     &  2.14\%      \\
        Colegio Pedro Mart\'inez V\'azquez (Maristas) &   339     &  1.95\%      \\
        \hline
        \multicolumn{3}{l}{Fuente: \citep{Seg2010}}
    \end{tabular}
\end{table}


\begin{table}[h]
    \centering
    \caption{Caracter\'isticas de los Principales Competidores}
    \label{tbl:CompetidoresDetalle}
    \footnotesize
    \begin{tabular}{l|c|c|c|c|c}
                                & Bachillerato de   &            &          & Universidad  & Nuestra   \\ 
        Característica          & la Universidad    & I.T.E.S.M. & CCPV     & Quetzalcóatl & Propuesta \\ 
        \hline
        \hline
        Imagen de Marca         & Baja              & Alta       & Media    & Baja         & Media     \\ 
        Precio                  & \$ 600            & \$ 6,000   & \$3,800  & \$ 1,500     & \$ 3,000  \\ 
        Orientación Técnica     & Alta              & Alta       & Media    & Alta         & Alta      \\ 
        Orientación Humanística & Media             & Baja       & Alta     & Baja         & Alta      \\ 
        Carreras Técnicas       & No                & No         & No       & Sí           & Sí        \\
        Presupuesto para Becas  & N/A               & 20-30 \%   & 20-30 \% & 20-30 \%     & 40-50 \%  \\
        \hline
        \multicolumn{6}{l}{Fuente: Comunidad Salesiana de Irapuato}
    \end{tabular} 
\end{table}


\clearpage
\section{Análisis FODA}
\label{sec:Neg:Foda}

El análisis FODA se muestra en el siguiente cuadro:

\begin{table}[h!]
    \centering
    \caption{Análisis FODA$^{/a}$ del Proyecto de la Escuela Salesiana}
    \label{tbl:Foda}
    \footnotesize
    \begin{tabular}{r|p{5in}}
    	\multicolumn{2}{c}{FORTALEZAS} \\
    	\hline
    	\hline
    	F-1 & Se cuenta con un inmueble con 40 salones, dos espacios para talleres, aulas para laboratorios, áreas administrativas, auditorio, campo de futbol, etc. \\
    	F-2 & Experiencia exitosa (SLP, Saltillo) \\
    	F-3 & Prestigio de la orden Salesianos de Don Bosco \\
    	F-4 & Interés por parte de algunos padres de la inspectoría por implementar el bachillerato tecnológico \\
    	F-5 & Un grupo de alrededor de 70 ex salesianos interesados y colaborando en el proyecto \\
    	F-6 & Ya se cuenta con las incorporaciones correspondientes ante la SEP \\
    	\hline
    	\multicolumn{2}{c}{DEBILIDADES} \\
    	\hline
    	\hline
    	D-1 & El P. Inspector aún no se decide por el proyecto \\
    	D-2 & Son pocos padres salesianos en Irapuato (menos de 6) \\
    	D-3 & La comunidad (sacerdotes, religiosas, laicos) tiene múltiples responsabilidades además de la escuela \\
    	D-4 & El \emph{staff} del proyecto no est\'a plenamente consolidado \\
    	\hline
    	\multicolumn{2}{c}{OPORTUNIDADES} \\
    	\hline
    	\hline
    	O-1 & Hay un fuerte impulso al desarrollo del corredor industrial de Guanajuato \\
    	O-2 & Por su posición geográfica, Irapuato, se está constituyendo en el centro de una conformación metropolitana, en la que para empezar se está integrando Salamanca \\
    	O-3 & Irapuato se puede convertir en un polo educativo, que dé servicio a varios municipios \\
    	O-4 & La cercanía con Querétaro coloca a Irapuato en situación vinculante Guanajuato-Querétaro \\
    	O-5 & El gobierno del estado está impulsando fuerte la educación media superior y superior \\
    	O-6 & Empresas grandes y fuertes están invirtiendo en la entidad \\
    	O-7 & El interés por los bachilleratos tecnológicos está creciendo \\
    	\hline
    	\multicolumn{2}{c}{AMENAZAS} \\
    	\hline
    	\hline
    	A-1 & La crisis económica actual podría obligar a los estudiantes a inscribirse a una escuela oficial o a abandonar los estudios \\
    	A-2 & Mercado competido dificulta la entrada de nuevos competidores \\
    	A-3 & La Universidad Quetzalcóatl ofrece más carreras técnicas y su precio general es más económico que el propuesto \\
    	A-4 & Existe el riesgo de no estar listos a tiempo para promover la escuela (febrero-abril) y que todo se retrase un año \\
    	\hline
    	\multicolumn{2}{l}{\footnotesize Fuente: Elaboración propia.} \\
        \multicolumn{2}{l}{$^{/a}$\footnotesize Más información sobre el análisis FODA en \citep{DAVID2003}.}
    \end{tabular}
\end{table}



\subsection{Estrategias Generales}

Considerando el escenario planteado en el análisis FODA, se proponen las siguientes estrategias generales:

\begin{itemize}
	\item Conformar un patronato integrado por miembros de la comunidad, ex-alumnos salesianos y empresarios interesados en el proyecto.
	\item Formar una red social a partir de los ex-alumnos salesianos que involucre a la Iglesia, simpatizantes, empresarios, funcionarios públicos, etc.; que sirva de soporte logístico para el resto de las estrategias.
	\item Promover el sistema pedagógico de Don Bosco ponendo énfasis en el tema de la prevención de la violencia.
	\item Realizar una campaña financiera con las empresas de la región para costear la inversión en infraestructura.
	\item Vincular a las empresas al proyecto orientando las carreras técnicas a las necesidade de la industria local.
	\item Apoyarse en la estructura de la diócesis y los grupos juveniles para promover la escuela.
	\item Apalancarse en el \emph{Know-How} de los salesianos para la elaboración de los planes y programas de estudio, así como su validación ante autoridades.
	\item Realizar una estrategia de comunicación que muestre simultáneamente las cualidades de la institución y la urgencia de educar a la población tanto en valores como técnicamente teniendo como eje la figura de Don Bosco.
\end{itemize}

\clearpage

\section{Estimaciones}
\label{sec:Neg:Estimaciones}

\subsection{Tamaño y Valor del Mercado}

La matrícula de bachillerato en 2009 es de 17,374 estudiantes, más los 830 jóvenes que no continúan sus estudios, tenemos un mercado de 18,204 personas\footnote{\citep{Seg2010}}.

El gasto promedio en colegiaturas en Guanajuato es de \$ 235.20\footnote{\citep{INEGI-2009-DGES-003}} para todos los niveles, con una desviación estándar \$ 689.30. Sin embargo, la misma encuesta informa que sólo el 13.2 \% de los estudiantes asiste al bachillerato.

Considerando esta información y que el costo mínimo a nivel bachillerato de las colegiaturas y en opinión de las personas del lugar\footnote{Entrevista del autor con el Pbro. Edmundo Morales, S.D.B., 16 de octubre de 2010}, la colegiatura promedio a considerar es de \$ 1,300.00 al mes. Por lo tanto el valor del mercado es de \$~23,665,200.00 mensual y \$~283,982,400.00 anual.

\subsection{Matrícula Esperada}

Las instalaciones tienen capacidad para aproximadamente 1000 alumnos, lo cual equivale a 5.5\% del mercado. Sin embargo, una escuela se va estructurando por generaciones; esto es, al principio sólo se atiende a los alumnos de primer grado, ellos conforman la primera generación. Al año siguiente, la primera generación de estudiantes pasa al segundo grado, una nueva generación se inscribe en el primero y así sucesivamente.

Por lo tanto, el primer año se espera 33\%, el segundo 67\% y al tercero 100\% del alumnado. A estos porcentajes se aplicarán tres escenarios: pesimista, intermedio e idealista. El escenario pesimista equivale a un tercio del total de estudiantes esperados, al escenario intermedio corresponden dos tercios y el escenario optimista significa tener toda la capacidad por grado escolar ocupada (ver cuadro \ref{tbl:MatriculaEsperada}). Para los escenarios intermedio y pesimista se considera seguir en crecimiento los años 4 y 5 de tal forma que al quinto se tendrían matrículas del 100\% y 67\% respectivamente.

\begin{table}[h]
    \centering
    \caption{Gasto promedio mensual en educación por integrante de familia}
    \label{tbl:EIGHG:gastos}
    \begin{tabular}{l|c|c|c|c}
        TIPO DE GASTO & MINIMO  & MAXIMO      & PROMEDIO  & DESVEST   \\
        \hline
        \hline
        Colegiaturas  & \$ 0.00 & \$ 7,900.00 & \$ 235.20 & \$ 689.30 \\
        Inscripción   & \$ 0.00 & \$ 6,500.00 & \$ 138.90 & \$ 560.40 \\
        Material      & \$ 0.00 & \$ 8,500.00 & \$ 213.00 & \$ 337.80 \\
        \hline
        \multicolumn{5}{l}{Fuente: \citep{INEGI-2009-DGES-003}}
    \end{tabular}
\end{table}


\begin{table}[h]
    \centering
    \caption{Tipo de Escuela a la que Asiste}
    \label{tbl:EIGHG:tipo-escuela}
    \begin{tabular}{l|c}
        \multicolumn{1}{c|}{TIPO ESCUELA} & ASISTENCIA \\
        \hline
        \hline
        Preprimaria       & 11.9\%     \\
        Primaria          & 46.2\%     \\
        Secundaria        & 19.6\%     \\
        Preparatoria      & 13.2\%     \\
        Profesional       &  8.1\%     \\
        Posgrado          &  0.6\%     \\
        Educación técnica &  0.4\%     \\
        \hline
        \multicolumn{2}{l}{Fuente: \citep{INEGI-2009-DGES-003}}
    \end{tabular}
\end{table}



\begin{table}[h]
    \centering
    \caption{Matrícula Esperada por año y Escenario}
    \label{tbl:MatriculaEsperada}
    \begin{tabular}{l|c|c|c|r|r|r}
                   &           & AÑO 1 & AÑO 2 & AÑO 3  & AÑO 4  & AÑO 5 \\
        ESCENARIO  & MATRICULA & 33\%  & 67\%  & 100\%  & 100\%  & 100\% \\
        \hline                                                  
        \hline                                                  
        Optimista  & 1         & 333   & 667   & 1000   & 1,000  & 1,000  \\
        Intermedio & 2/3       & 222   & 445   & 667    & 890    & 1,000  \\
        Pesimista  & 1/3       & 110   & 220   & 330    & 500    & 667    \\
        \hline
        \multicolumn{7}{l}{\footnotesize Fuente: Elaboración propia, 2010.}
    \end{tabular}
\end{table}



\clearpage
