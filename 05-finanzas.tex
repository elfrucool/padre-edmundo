\chapter{Administración de Finanzas}
\label{cap:Admin:Finanzas}

\section{Punto de Equilibrio}

Mediante el cálculo del punto de equilibrio se pretende determinar el alumnado mínimo necesario para que la institución no incurra en pérdidas\footnote{Consúltense las siguientes fuentes para esta sección: \citep{Van2003fundamentos, novoa2008finanzas, bodie2003finanzas}}.

El punto de equilibrio se obtiene algebráicamente de la siguiente forma:

$$
	\begin{array}{rcl}
		P \cdot U &=& C_{vu} \cdot U + C_f \\
		\left( P - C_{vu} \right) \cdot U &=& C_f \\
		U &=& \frac{C_f}{P - C_{vu}}
	\end{array}
$$

Donde:

\begin{description}
	\item[$P$] Precio de venta unitario
	\item[$U$] Unidades del punto de equilibrio
	\item[$C_{vu}$] Costo variable unitario
	\item[$C_f$] Costo fijo
\end{description}

El cuadro \ref{tbl:PuntoEquilibrio} muestra el punto de equilibrio, a excepción del primer año, el resto del periodo estudiado se rebasa este punto de equilibrio; lo cual puede corroborarse con las utilidades en el estado de resultados.

\begin{table}
    \caption{Punto de Equilibrio}
    \label{tbl:PuntoEquilibrio}
    \centering
    \footnotesize
    \begin{tabular}{@{\hspace{1mm}}l@{\hspace{1mm}}*{5}{|@{\hspace{1mm}}r@{\hspace{1mm}}}}
	%-------------------------------------------------- \\
	\hline
	\multicolumn{1}{c|}{CONCEPTO} &
	    \multicolumn{1}{c|}{AÑO 1} &
	    \multicolumn{1}{c|}{AÑO 2} &
	    \multicolumn{1}{c|}{AÑO 3} &
	    \multicolumn{1}{c|}{AÑO 4} &
	    \multicolumn{1}{c}{AÑO 5} \\
	\hline
	\hline
	%-------------------------------------------------- \\
	Ingreso por Unidad	&	2,100.00	&	2,100.00	&	2,100.00	&	1,800.00	&	1,800.00 \\
	Mensual &&&&& \\
	Costos Variables	&	22.72	&	23.10	&	22.20	&	23.06	&	25.76 \\
	por Unidad Mensual &&&&& \\
	Margen de Contribución	&	2,077.28	&	2,076.90	&	2,077.80	&	1,776.94	&	1,774.24 \\
	por Unidad Mensual &&&&& \\
	%-------------------------------------------------- \\
	\hline
	Costos Fijos Anuales	&	6,038,930.64	&	8,802,880.79	&	13,190,506.27	&	15,301,561.60	&	16,755,531.14 \\
	Margen de Contribución	&	24,927.30	&	24,922.85	&	24,933.63	&	21,323.31	&	21,290.92 \\
	por Unidad Anual &&&&& \\
	\hline
	\hline
	PUNTO DE EQUILIBRIO	&	242	&	353	&	529	&	718	&	787 \\
	%-------------------------------------------------- \\
	\hline
	\multicolumn{6}{l}{\footnotesize Fuente: Elaboración Propia, 2010.}
    \end{tabular}
\end{table}


\section{Utilidades}

Con la finalidad de comparar mejor el punto de equilibrio, se presenta a continuación la parte que se refiere a las utilidades juntamente con la matrícula esperada para cada año (cuadro \ref{tbl:Utilidades}).

El ejercicio del primer año otorga pérdidas, sin embargo, el resto de los años traen utilidades y lo realizan en forma creciente. Esto a pesar de que el precio de la colegiatura no se incrementa con la inflación y el porcentaje de becas se incrementa al incrementar la matrícula estudiantil.

\begin{table}[h]
    \caption{Utilidades}
    \label{tbl:Utilidades}
    \centering
    \footnotesize
    \begin{tabular}{l*{5}{|r}}
	\multicolumn{1}{c|}{CONCEPTO} &
	    \multicolumn{1}{c|}{1} &
	    \multicolumn{1}{c|}{2} &
	    \multicolumn{1}{c|}{3} &
	    \multicolumn{1}{c|}{4} &
	    \multicolumn{1}{c}{5} \\
	\hline
	\hline
	Utilidad del Ejercicio	&	-1,482,058.59 	&	 669,605.03 	&	 1,296,961.29 	&	 1,419,653.81 	&	 2,046,763.53 \\
	Matrícula	&	222	&	445	&	667	&	290	&	1000 \\
	\hline
	\multicolumn{6}{l}{\footnotesize Fuente: Elaboración Propia, 2010.}
    \end{tabular}
\end{table}



\clearpage
\section{Flujo de Efectivo}

A continuación se presentan dos instrumentos financieros con amplia relación: el estado de origen y aplicación de los recursos y el flujo neto de efectivo.

\subsection{Origen y Aplicación de los Recursos}

El estado de origen y aplicación de los recursos describe la forma en que una empresa obtiene y utiliza los fontos monetarios, nos permite evaluar si el proyecto tendrá el efectivo necesario para hacer frente a sus obligaciones, entendidas éstas como cubrir los costos y pagar las deudas. El cuadro \ref{tbl:OrigenAplicacion} muestra cómo este proyecto tiene la liquidez necesaria, lo cual se puede ver en el último renglón donde todas las cifras son positivas.

En términos de liquidez, puede observarse que los primeros dos años y el cuarto son los más difíciles. De los primeros dos, la causa se debe a que en el primer año se tienen pérdidas por no haber alcanzado el punto de equilibrio, del cuarto año, la razón es que por una parte se incrementó el organigrama y por otro lado disminuyen los ingresos por estudiante debido a un mayor porcentaje de becas otorgado como consecuencia de tener una matrícula estudiantil mayor.

Por último, aclarar que la cuenta de saldo en caja y bancos\footnote{Parte del activo circulante} proviene de este instrumento; y es precisamente el rubro de \emph{saldo final de efectivo}.

\begin{table}[h]
    \caption{Estado de Origen y Aplicación de Recursos}
    \label{tbl:OrigenAplicacion}
    \centering
    \scriptsize
    \begin{tabular}{@{\hspace{1mm}}r@{\hspace{1mm}}l@{\hspace{1mm}}*{6}{|@{\hspace{1mm}}r@{\hspace{1mm}}}}
	& \multicolumn{1}{c|}{CONCEPTO} &
	    \multicolumn{1}{c|}{AÑO 0} &
	    \multicolumn{1}{c|}{AÑO 1} &
	    \multicolumn{1}{c|}{AÑO 2} &
	    \multicolumn{1}{c|}{AÑO 3} &
	    \multicolumn{1}{c|}{AÑO 4} &
	    \multicolumn{1}{c}{AÑO 5} \\
    \hline
    \hline
      & SALDO INICIAL DE EFECTIVO             &  12,011,099.40 	&	 6,042,800.00 	&	 2,319,568.17 	&	 2,399,147.14 	&	 2,870,784.80 	&	 2,708,235.54  \\
    \hline
      & Ingresos Para el Año n+1              & 0.0 	&	 5,602,800.00 	&	 11,205,600.00 	&	 16,808,400.00 	&	 19,224,000.00 	&	 21,600,000.00  \\
    + & Entradas de efectivo                  &  0.0 	&	 5,602,800.00 	&	 11,205,600.00 	&	 16,808,400.00 	&	 19,224,000.00 	&	 21,600,000.00  \\
    \hline
    = & DISPONIBILIDAD DE EFECTIVO            &  12,011,099.40 	&	 11,645,600.00 	&	 13,525,168.17 	&	 19,207,547.14 	&	 22,094,784.80 	&	 24,308,235.54  \\
    \hline
      & Total Costos Variables                &  0.0 	&	 60,812.00 	&	 123,333.00 	&	 177,665.67 	&	 246,249.99 	&	 309,078.93  \\
      & Total Costos Fijos & 0.0 	&	 3,426,750.24 	&	 5,877,168.30 	&	 8,138,395.70 	&	 10,118,187.02 	&	 11,437,362.89  \\
      & (Sin Depreciación) &&&&& \\
      & Total de Gastos de Operación          & 0.0 	&	 3,172,596.05 	&	 3,506,128.13 	&	 5,662,915.21 	&	 5,824,355.27 	&	 5,809,148.94  \\
      & Total Gastos financieros              &  0.0 	&	 312,644.30 	&	 257,491.49 	&	 195,343.92 	&	 125,314.48 	&	 46,403.55  \\
      & Pago de ISR y PTU                     &  0.0 	&	 0.0 	&	 547,762.05 	&	 1,000,950.21 	&	 1,105,759.43 	&	 1,519,242.16  \\
      & Impuesto al Depósido en Efectivo      & 0.0 	&	 112,056.00 	&	 224,112.00 	&	 336,168.00 	&	 384,480.00 	&	 432,000.00  \\
      & Pago de crédito (Capital)             & 0.0 	&	 434,873.25 	&	 490,026.06 	&	 552,173.63 	&	 622,203.07 	&	 701,113.99  \\
      & Inversión en Terrenos                 & 3,441,776.45 	&	 0.0 	&	 0.0 	&	 0.0 	&	 0.0 	&	 0.0  \\
      & Inversión en Edificios                & 2,422,632.95 	&	 0.0 	&	 0.0 	&	 0.0 	&	 0.0 	&	 0.0  \\
      & Inversión en Mobiliario y Equipos     & 103,890.00 	&	 1,806,300.00 	&	 100,000.00 	&	 273,150.00 	&	 960,000.00 	&	 0.0  \\
    \hline
    - & SALIDAS DE EFECTIVO                   &  5,968,299.40	&	9,326,031.83	&	11,126,021.03	&	16,336,762.34	&	19,386,549.25	&	20,254,350.47 \\
    \hline
    \hline
    = & SALDO FINAL DE EFECTIVO               &  6,042,800.00	&	2,319,568.17	&	2,399,147.14	&	2,870,784.80	&	2,708,235.54	&	4,053,885.08 \\
    \hline
    \multicolumn{8}{l}{\footnotesize Fuente: Elaboración Propia, 2010.}
    \end{tabular}
\end{table}



\subsection{Flujo Neto de Efectivo}
\label{sub:FNE}

El flujo neto de efectivo (FNE) se parece al estado de origen y aplicación de los recursos en que describe cómo se obtienen y utilizan los recursos en una organización.\footnote{La siguiente explicación corresponde al flujo neto de efectivo acumulado.}

Sin embargo tiene una diferencia capital: el monto de la inversión inicial se coloca con signo negativo, lo mismo que sucesivas inversiones en caso de existir. En el contexto de evaluación de proyectos, esto permite evaluar la capacidad de la empresa para recuperar lo invertido. Al inversionista (y al accionista) le interesa saber en qué medida se recupera el capital invertido en comparación con otras ofertas. Es la materia prima de la evaluación final de un proyecto. En el capítulo \ref{cap:Evaluacion:Financiera} (página \pageref{cap:Evaluacion:Financiera}) se retomará este tema precisamente para validar la viabilidad del bachillerato.

Conviene, llegados a este punto, realizar una aclaración: aunque el monto total de la inversión inicial es de \$12,011,099.40, los \$6,210,709.40 que conforman el inmueble (terreno y edificios) y mobiliario existentes son prescindibles en este análisis; la razón es que se trata de un capital del que no hay accionistas que pedirán cuentas y al que no se le dará otro uso si no es la escuela. En ese sentido es un \emph{capital muerto}.

Esta consideración lleva a plantear dos escenarios: con inmueble y mobiliario existentes y sin ellos. Los inversionistas para este proyecto que reclamarán resultados sobre sus aportaciones son las empresas patrocinadoras.

Por esta razón, el cuadro \ref{tbl:FNE} presenta al final un resumen con los resultados más relevantes cuando se prescinde de la infraestructura con la que se cuenta actualmente. Los datos de este resumen son los utilizados en el capitulo \ref{cap:Evaluacion:Financiera} para realizar la evaluación financiera.

\begin{table}
    \caption{Flujo Neto de Efectivo}
    \label{tbl:FNE}
    \centering
    \scriptsize
    \begin{tabular}{@{\hspace{1mm}}r@{\hspace{1mm}}|@{\hspace{1mm}}l@{\hspace{1mm}}*{6}{|@{\hspace{1mm}}r@{\hspace{1mm}}}}
	%--------------------------------------------------
	& \multicolumn{1}{c|}{CONCEPTO} &
	    \multicolumn{1}{c|}{AÑO 1} &
	    \multicolumn{1}{c|}{AÑO 2} &
	    \multicolumn{1}{c|}{AÑO 3} &
	    \multicolumn{1}{c|}{AÑO 4} &
	    \multicolumn{1}{c}{AÑO 5} \\
	%--------------------------------------------------
	\hline
	\hline
	& \multicolumn{7}{c}{FLUJO NETO DE EFECTIVO CONSIDERANDO \emph{CAPITAL MUERTO}} \\
	\hline
	\hline
	1	&	Inversion Inicial                & -12,011,099.40 & & & & & \\
	\hline
	2	&	SALDO INICIAL                    &                & -12,011,099.40 	&	-12,932,742.34 	&	-11,682,721.66 	&	-9,774,955.72 	&	-7,714,321.22  \\
	\hline
	3	&	Ingresos Netos                   &                &  5,602,800.00 	&	 11,205,600.00 	&	 16,808,400.00 	&	 19,224,000.00 	&	 21,600,000.00  \\
	\hline
	4	&	TOTAL DE INGRESOS                &                & -6,408,299.40 	&	-1,727,142.34 	&	 5,125,678.34 	&	 9,449,044.28 	&	 13,885,678.78  \\
	\hline
	\hline
	5	&	\multicolumn{7}{l}{COSTOS VARIBALES} \\
	\hline
	6	&	Papelería                        &                &  12,372.00 	&	 23,373.00 	&	 35,840.07 	&	 48,990.69 	&	 57,323.27  \\
	7	&	Material Didáctico               &                &  13,440.00 	&	 26,460.00 	&	 42,600.60 	&	 58,344.30 	&	 69,429.72  \\
	8	&	Material talleres                &                &  35,000.00 	&	 73,500.00 	&	 99,225.00 	&	 138,915.00 	&	 182,325.94  \\
	\hline
	9	&	TOTAL COSTOS VARIBLES            &                &  60,812.00 	&	 123,333.00 	&	 177,665.67 	&	 246,249.99 	&	 309,078.93  \\
	\hline
	10	&	UTILIDAD BRUTA                   &                & -6,469,111.40 	&	-1,850,475.34 	&	 4,948,012.67 	&	 9,202,794.29 	&	 13,576,599.85  \\
	\hline
	\hline
	11	&	\multicolumn{7}{l}{COSTOS FIJOS} \\
	\hline
	12	&	Servicios subcontratados         &                &  836,000.00 	&	 1,570,800.00 	&	 1,649,340.00 	&	 1,731,807.00 	&	 1,818,397.35  \\
	13	&	Servicios (agua, luz, telefono)  &                &  520,000.00 	&	 546,000.00 	&	 573,300.00 	&	 601,965.00 	&	 632,063.25  \\
	14	&	Sueldos Directos                 &                &  2,070,750.24 	&	 3,760,368.30 	&	 5,915,755.70 	&	 7,784,415.02 	&	 8,986,902.29  \\
	\hline
	15	&	TOTAL COSTOS FIJOS               &                &  3,426,750.24 	&	 5,877,168.30 	&	 8,138,395.70 	&	 10,118,187.02 	&	 11,437,362.89  \\
	\hline
	16	&	MARGEN DE UTILIDAD               &                & -9,895,861.64 	&	-7,727,643.64 	&	-3,190,383.03 	&	-915,392.73 	&	 2,139,236.97  \\
	\hline
	\hline
	17	&	\multicolumn{7}{l}{GASTOS DE OPERACIÓN} \\
	\hline
	18	&	\multicolumn{7}{l}{GASTOS DE ADMINISTRACION} \\
	\hline
	19	&	Sueldos Indirectos               &                &  1,195,708.80 	&	 1,225,601.52 	&	 2,284,075.56 	&	 2,341,177.45 	&	 2,399,706.89  \\
	20	&	Sueldos Administrativos          &                &  1,236,471.60 	&	 1,511,110.97 	&	 2,569,585.01 	&	 2,633,824.63 	&	 2,699,670.25  \\
	\hline
	21	&	\multicolumn{7}{l}{GASTOS DE VENTA} \\
	\hline
	22	&	Publicidad                       &                &  180,000.00 	&	 189,000.00 	&	 198,450.00 	&	 208,372.50 	&	 218,791.13  \\
	\hline
	23	&	TOTAL DE GASTOS     &                &  2,612,180.40 	&	 2,925,712.49 	&	 5,052,110.57 	&	 5,183,374.58 	&	 5,318,168.26  \\
		&	DE OPERACIÓN & & & & & \\
	\hline
	24	&	\multicolumn{7}{l}{GASTOS Y PRODUCTOS FINANCIEROS} \\
	\hline
	25	&	Gastos Financieros               &                &  312,644.30 	&	 257,491.49 	&	 195,343.92 	&	 125,314.48 	&	 46,403.55  \\
	\hline
	26	&	TOTAL GASTOS         &                &  312,644.30 	&	 257,491.49 	&	 195,343.92 	&	 125,314.48 	&	 46,403.55  \\
		&	FINANCIEROS & & & & & \\
	\hline
	27	&	UTILIDAD DE OPERACIÓN            &                & -12,820,686.34 	&	-10,910,847.61 	&	-8,437,837.51 	&	-6,224,081.79 	&	-3,225,334.85  \\
	\hline
	\hline
	28	&	\multicolumn{7}{l}{OTROS GASTOS Y PRODUCTOS} \\
	\hline
	29	&	UTILIDAD ANTES      &                & -12,820,686.34 	&	-10,910,847.61 	&	-8,437,837.51 	&	-6,224,081.79 	&	-3,225,334.85  \\
		&	DE IMPUESTOS & & & & & \\
	\hline
	\hline
	30	&	Provision de ISR                 &                &  0.0 	&	 403,614.14 	&	 737,542.26 	&	 814,770.11 	&	 1,119,441.59  \\
	31	&	Provision de PTU                 &                &  0.0 	&	 144,147.91 	&	 263,407.95 	&	 290,989.32 	&	 399,800.57  \\
	32	&	Impuesto al Depósido &                &  112,056.00 	&	 224,112.00 	&	 336,168.00 	&	 384,480.00 	&	 432,000.00  \\
		& en Efectivo & & & & & \\
	\hline
	33	&	TOTAL IMPUESTOS                  &                &  112,056.00 	&	 771,874.05 	&	 1,337,118.21 	&	 1,490,239.43 	&	 1,951,242.16  \\
	\hline
	\hline
	34	&	FLUJOS NETOS         & -12,011,099.40 & -12,932,742.34 	&	-11,682,721.66 	&	-9,774,955.72 	&	-7,714,321.22 	&	-5,176,577.01  \\
		&	DE EFECTIVO & & & & & \\
	\hline
	\hline
	& \multicolumn{7}{c}{FLUJO NETO DE EFECTIVO SIN CONSIDERAR \emph{CAPITAL MUERTO}} \\
	%--------------------------------------------------
	\hline
	\hline
	1	&	Inversion Inicial                & -5,800,390.00 & & & & & \\
	\hline
	2	&	SALDO INICIAL                    &               & -5,800,390.00 	&	-6,722,032.94 	&	-5,472,012.26 	&	-3,564,246.32 	&	-1,503,611.82  \\
	\hline
	34	&	FLUJOS NETOS         & -5,800,390.00 & -6,722,032.94 	&	-5,472,012.26 	&	-3,564,246.32 	&	-1,503,611.82 	&	 1,034,132.39  \\
		&	DE EFECTIVO & & & & & \\
	\hline
	\multicolumn{8}{l}{\footnotesize Fuente: Elaboración Propia, 2010.}
    \end{tabular}
\end{table}


\section{Activos y Pasivos}
\label{sec:ActivosPasivos}

En esta sección se comentan algunos aspectos relevantes sobre la estructura de activos y pasivos previstos en el bachillerato.

El cuadro \ref{tbl:Activo:Resumen} presenta las cuentas del activo y el pasivo sobre las que versará la discusión.

Las principales observaciones sobre las cuentas de activo y pasivo relevantes son:

\begin{description}
	\item[Caja y Bancos] Con recuperación creciente (a excepción del año 4 debido al aumento de porcentaje de becas y la inversión en activo fijo) en el periodo de 5 años no alcanza a empatar la inversión inicial de su rubro.
	\item[Cuentas por Cobrar] Se trata de una \emph{cuenta virtual} consecuencia de la depreciación. La razón es que no se renueva el activo fijo en el mismo ritmo que se deprecia.
	\item[Activo Fijo] El activo fijo incrementa su valor en los años 1 y 4 debido a las fuertes inversiones realizadas esos años. El resto de los años disminuye el valor debido a la depreciación.
	\item[Total Activo] Después de una caída el primer año, el total del activo se incrementa y para el quinto año es mayor al activo inicial.
	\item[Pasivo: Crédito] El pago del crédito concluye en el año 5. De momento no se tiene proyectado solicitar un préstamo similar los siguientes cinco años.
\end{description}

Considerando estas observaciones, la conclusión apunta en el sentido de que este proyecto se percibe más sólido financieramente en su segundo lustro de operación: sin una carga financiera considerable y con utilidades como las del último año se perfila hacia la consolidación. Convendría realizar un estudio considerando una ventana de tiempo de 10 años que mostrara con mayor claridad su viabilidad y conveniencia financiera.

\begin{table}
    \caption{Cuentas Relevantes del Activo y el Pasivo}
    \label{tbl:Activo:Resumen}
    \centering
    \scriptsize
    \begin{tabular}{@{\hspace{1mm}}l@{\hspace{1mm}}*{6}{|@{\hspace{1mm}}r@{\hspace{1mm}}}}
%--------------------------------------------------
    \multicolumn{1}{c|}{} &
	\multicolumn{6}{c}{AÑO}	\\
    \cline{2-7}
    \multicolumn{1}{c|}{CONCEPTO} &
	\multicolumn{1}{c|}{0} &
	\multicolumn{1}{c|}{1} &
	\multicolumn{1}{c|}{2} &
	\multicolumn{1}{c|}{3} &
	\multicolumn{1}{c|}{4} &
	\multicolumn{1}{c}{5} \\
%--------------------------------------------------
    \hline
    \hline
    ACTIVO CIRCULANTE \\
    Caja y bancos                    &	 6,042,800.00 	&	 2,319,568.17 	&	 2,399,147.14 	&	 2,870,784.80 	&	 2,708,235.54 	&	 4,053,885.08  \\
    Cuentas por cobrar               &	 0.0 	&	 560,415.65 	&	 1,140,831.30 	&	 1,751,635.94 	&	 2,392,616.63 	&	 2,883,597.32  \\
    \hline
    ACTIVO FIJO                      &		&		&		&		&		&	 \\
    TOTAL ACTIVO FIJO                &	 5,968,299.40 	&	 7,214,183.75 	&	 6,733,768.11 	&	 6,396,113.46 	&	 6,715,132.77 	&	 6,224,152.08  \\
    \hline
    TOTAL ACTIVO                     &	 12,011,099.40 	&	 10,094,167.57 	&	 10,273,746.54 	&	 11,018,534.20 	&	 11,815,984.94 	&	 13,161,634.48  \\
    \hline
    PASIVO                           &		&		&		&		&		&	 \\
    \hline
    Crédito                          &	 2,800,390.00 	&	 2,365,516.75 	&	 1,875,490.69 	&	 1,323,317.06 	&	 701,113.99 	&	 0.00  \\
%--------------------------------------------------
	\hline
	\multicolumn{7}{l}{\footnotesize Fuente: Elaboración Propia, 2010.}
    \end{tabular}
\end{table}



\section{Financiamiento Requerido}
\label{sec:Financiamiento}

Para hacer realidad este proyecto se necesita un financiamiento considerado desde dos aspectos: por un lado, la solicitud de un préstamo a una tasa de interés conveniente; por otra parte, la búsqueda de patrocinios y donativos.

En esta sección se hablará sobre el préstamo y en la siguiente sobre los patrocinios y donativos.

\subsection{Características del Préstamo}

Se propone la solicitud de un financiamiento ante la Secretaría de Economía por ofrecer bajas tasas de interés en comparación con entidades privadas (cuadro \ref{tbl:Credito}).

Las condiciones del financiamiento son:

\begin{itemize}
	\item Monto del crédito: \$2,800,390.00 M.N.
	\item Interés anual: 12\%.
	\item Periodo de recuperación: 5 años.
	\item Número de pagos: 60 pagos mensuales.
	\item Pago mensual: \$62,293.13 M.N.
	\item No se ofrece periodo de gracia.
\end{itemize}

Las áreas principales de aplicación son: capital de trabajo y remodelaciones.

\begin{table}
    \caption{Tabla de Amortización Anual del Crédito}
    \label{tbl:Credito}
    \centering
    \footnotesize
    \begin{tabular}{c*{5}{|r}}
    %--------------------------------------------------
	&   \multicolumn{1}{c|}{SALDO} &
	    \multicolumn{1}{c|}{PAGO DE} &
	    \multicolumn{1}{c|}{PAGO} \\
	\multicolumn{1}{c|}{AÑO} &
	    \multicolumn{1}{c|}{INICIAL} &
	    \multicolumn{1}{c|}{INTERÉS} &
	    \multicolumn{1}{c|}{PRINCIPAL} &
	    \multicolumn{1}{c|}{PAGO TOTAL} &
	    \multicolumn{1}{c}{SALDO FINAL} \\
	\hline
	\hline
	1	&	2,800,390.00	&	312,644.30	&	434,873.25	&	747,517.55	&	2,365,516.75 \\
	2	&	2,365,516.75	&	257,491.49	&	490,026.06	&	747,517.55	&	1,875,490.69 \\
	3	&	1,875,490.69	&	195,343.92	&	552,173.63	&	747,517.55	&	1,323,317.06 \\
	4	&	1,323,317.06	&	125,314.48	&	622,203.07	&	747,517.55	&	701,113.99 \\
	5	&	701,113.99	&	46,403.55	&	701,113.99	&	747,517.55	&	0.00 \\
    %--------------------------------------------------
    \hline
    \multicolumn{6}{l}{\footnotesize Fuente: Elaboración Propia, 2010.}
    \end{tabular}
\end{table}


\section{Contribución de Capital}
\label{sec:Capital}

Como se señaló en el cuadro \ref{tbl:ActivoFijo} (página \pageref{tbl:ActivoFijo}), se cuenta con una infraestructura equivalente a \$6,210,709.40 M.N. entre terreno, edificios y mobiliario existente.

Adicionalmente, se desea obtener un patrocinio inicial equivalente a \$3,000,000.00 tanto para iniciar el proyecto como para ofrecer becas.

La posibilidad de conseguir esos patrocinios (en dinero o en especie) se basa en las experiencias exitosas de Saltillo y San Luis Potosí, donde incluso se donaron terrenos para esas escuelas (subsección \ref{sub:Patrocinios}, página \pageref{sub:Patrocinios}).

En total, el capital inicial es de \$9,210,709.40 M.N.; sin embargo, para efectos de evaluación financiera se debe considerar que las instalaciones existentes actualmente son \emph{capital muerto}, es decir, de no usarse para el bachillerato no se usarían para ninguna otra inversión\footnote{Subsección \ref{sub:FNE}, página \pageref{sub:FNE}}.

\section{Recursos Colaterales}
\label{sec:RecursosColaterales}

Adicionalmente a las colegiaturas, se pretende obtener de forma permanente donativos y patrocinios (ya sea en efectivo o en especie) que abaraten los costos e incrementen el fondo de becas.\footnote{Sin embargo, la estrategia para obtener dicho patrocinio está fuera del alcance del presente trabajo y la corrida financiera no contempla ninguna fuente alterna salvo la mencionada en la inversión inicial (ver \ref{sub:intro:Limitaciones} <<\nameref{sub:intro:Limitaciones}>> en la página \pageref{sub:intro:Limitaciones}.}

Para esto, se tienen previstas diferentes modalidades de patrocinios, entre otras:

\begin{itemize}
	\item Adopta un estudiante.
	\item Adopta un profesor.
	\item Patrocinio para equipar la escuela (equipos de cómputo y maquinaria de taller).
	\item Patrocinio para insumos de la escuela: papelería, material de talleres.
	\item Paquete especial para hijos de los trabajadores de la empresa.
	\item Apoyo a los estudiantes talentosos.
\end{itemize}

Similar a la pirámide de necesidades del mercado\footnote{Sección \ref{sec:Clientes}, página \pageref{sec:Clientes}}, se propone una pirámide de donadores de la siguiente forma:

\begin{itemize}
	\item Dos o tres patrocinadores grandes que cubran lo equivalente a 30 estudiantes cada uno.
	\item Diez a veinte patrocinadores medianos que cubran lo equivalente a 10 estudiantes cada uno.
	\item Cincuenta a cien patrocinadores pequeños que bequen a un estudiante cada uno.
	\item Apoyo general de la comunidad mediante aportaciones voluntarias en efectivo o especie.
\end{itemize}

Otras actividades para obtener recursos o ahorros adicionales son:

\begin{itemize}
	\item Profesores honorarios que donen algunas horas de su tiempo a la semana dando clase sin cobrar.
	\item Cooperativa escolar.
	\item Servicios a las empresas por parte de los estudiantes (programa de becarios).
	\item Conseguir apoyo de fundaciones (e.g. bécalos).
\end{itemize}
