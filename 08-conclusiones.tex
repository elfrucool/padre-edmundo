\chapter*{Conclusiones}
\addcontentsline{toc}{chapter}{Conclusiones}
\label{ch:Conclusiones}

\begin{quote}
<<\textit{Todos los hombres, de cualquier raza, condición y edad, en cuanto participantes de la dignidad de la persona, tienen el derecho inalienable de una educación, que responda al propio fin, al propio carácter; al diferente sexo, y que sea conforme a la cultura y a las tradiciones patrias, y, al mismo tiempo, esté abierta a las relaciones fraternas con otros pueblos a fin de fomentar en la tierra la verdadera unidad y la paz. Mas la verdadera educación se propone la formación de la persona humana en orden a su fin último y al bien de las varias sociedades, de las que el hombre es miembro y de cuyas responsabilidades deberá tomar parte una vez llegado a la madurez}>> \citep{GRED1965}.
\end{quote}

En este apartado se confrontarán los objetivos general y específicos con los resultados obtenidos a lo largo del presente trabajo; posteriormente se realizarán algunas sugerencias para que el proyecto se lleve a la práctica exitosamente.

\pagebreak
\section*{¿Se Cumplieron los Objetivos?}

La respuesta a esta pregunta se detallará revisando objetivo por objetivo comenzando por el objetivo general.

\subsection*{¿Se Cumplió el Objetivo General?}

%El objetivo general planteado en el presente trabajo es: <<verificar la viabilidad financiera y social de la Escuela Salesiana>>\footnote{Subsecci\'{o}n \ref{sub:ObjetivoGeneral}, p\'{a}gina \pageref{sub:ObjetivoGeneral}}.

El objetivo general planteado en el presente trabajo es: <<Verificar la viabilidad financiera de la Escuela Salesiana mediante el análisis de su corrida financiera para así decidir la puesta en marcha del proyecto>>\footnote{Subsección \ref{sub:ObjetivoGeneral}, página \pageref{sub:ObjetivoGeneral}}.

De acuerdo con la investigación realizada, el proyecto resulta financieramente viable pues rebasa el punto de equilibrio y mantiene unas finanzas lo suficientemente saludables como para hacer frente a sus obligaciones de corto y largo plazo, tal como se muestra en el estado de resultados\footnote{Cuadro \ref{tbl:EstadoResultados} página \pageref{tbl:EstadoResultados}.} y los indicadores financieros favorables (capítulo \ref{cap:AnalsisFinanciero}).

Sin embargo, el proyecto no resulta atractivo para inversionistas, entidades financieras o patrocinadores, razón por la cual, la estrategia para conseguir el financiamiento resulta ser crítica; esto puede verificarse en los cálculos del capítulo \ref{cap:Evaluacion:Financiera}, por ejemplo: el valor presente neto es negativo en aproximadamente \$2.5 millones de pesos y la tasa interna de retorno es apenas superior al 4\%.

%Como negocio, el proyecto no resulta atractivo; aunque una vez iniciado tiene la capacidad para sostenerse a sí mismo (estado de resultados, estado de origen y aplicación de los recursos, balance general, razones financieras); no tiene la capacidad de otorgar un valor al inversionista proporcional al riesgo en que incurre (valor presente neto, tasa interna de retorno, ROI, etc.).

\subsection*{¿Se cumplieron los objetivos específicos?}

A continuación se listan los objetivos específicos señalando si se cumplieron o no:

\begin{itemize}
	\item \emph{Determinar el monto total de la inversión inicial, mediante el análisis de las necesidades de liquidez e infraestructura, para así establecer la situación patrimonial inicial de la Escuela.}
	
		Se pudo determinar el monto total de la inversión inicial, siendo éste de: \$ 11,579,099.40.\footnote{Ver la subsección \ref{sub:oper:InversionInicial} en la página \ref{sub:oper:InversionInicial}.}

	\item \emph{Determinar las fuentes para el financiamiento inicial mediante la diferenciación entre el financiamiento patrocinado y la adquisición de deuda para precisar las obligaciones de largo plazo (pasivos) a cubrir.}

		El objetivo se cumplió, toda vez que el financiamiento inicial está compuesto por \$ 6,270,709.4 de activo fijo existente, \$ 3,000.000.00 de patrocinio y \$ 2,308,390.00 del crédito bancario.\footnote{Ver la subsección \ref{sub:oper:InversionInicial} en la página \ref{sub:oper:InversionInicial}.}

	\item \emph{Verificar la viabilidad financiera cuidando que siempre se rebase el punto de equilibrio, igualmente, garantizar que las razones financieras indiquen un resultado favorable para así asegurar que la Escuela puede \emph{valerse por sí misma}}.

		Este objetivo también se cumplió como puede verse en el estado de resultados, \footnote{Cuadro \ref{tbl:EstadoResultados} página \pageref{tbl:EstadoResultados}.} y en los indicadores financieros favorables. \footnote{Capítulo \ref{cap:AnalsisFinanciero} página \pageref{cap:AnalsisFinanciero}}

	\item \emph{Determinar la cobertura máxima de becas expresada en porcentaje sobre el total de los ingresos mediante el ajuste entre los egresos y los ingresos para así medir a cuántos estudiantes puede beneficiar la escuela si considera como fuentes de ingreso únicamente las de la inversión inicial y las colegiaturas.}

		Se cumplió el objetivo, toda vez que se determinó para diferentes niveles de ocupación de la escuela, los porcentajes máximos que garantizan el correcto funcionamiento de la escuela; siendo éstos porcentajes: 20\%, 30\% y 40\% para los niveles de ocupación menor a un tercio, entre uno y dos tercios y finalmente mas de dos tercios respectivamente.

	\item \emph{Identificar el punto de equilibrio considerando exclusivamente la inversión inicial y las aportaciones de los estudiantes (colegiaturas) para así evaluar la capacidad de la escuela para ofrecer becas.}

		La realización del objetivo fue exitosa, toda vez que se determinó el punto de equilibrio como consta en el cuadro \ref{tbl:PuntoEquilibrio} de la página \pageref{tbl:PuntoEquilibrio}.

	\item \emph{Verificar si el proyecto, así concebido, es además, rentable desde el punto de vista lucrativo, a través del análisis de las variables de evaluación financiera,lo cual ayudará a la comunidad salesiana de Irapuato a definir su estrategia para la obtención del financiamiento que requiere}.

	Este objetivo no se cumplió, ya que, como se muestra en el capítulo \ref{cap:Evaluacion:Financiera}, las variables de la evaluación financiera resultaron desfavorables.

\end{itemize}

%\section*{Las tres tensiones\footnote{Alusión al título del libro de FABARO, K.; Ed. Granica; 2008}}
\section*{Las tres tensiones\footnote{Alusión al título del libro de \citep{FABARO2008}}}

Esto origina una tensión entre diferentes polos de interés que es preciso mencionar:

\begin{itemize}
	\item Ser un proyecto atractivo para patrocinadores y entidades financieras.
	\item Cumplir con una función social de ayuda a jóvenes de escasos recursos.
	\item Retribuir justamente al personal que labora ahí y retener el talento.
\end{itemize}

Visto desde el esquema tradicional de la empresa, no es posible atender estos tres polos; el crecimiento de uno solo va en decremento de los otros dos. A estos tres polos pueden añadirse otros como el crecimiento o la calidad educativa.

Desde sus comienzos este proyecto nunca aspiró a ser un negocio
con finalidad lucrativa.
%para unos accionistas.
Bastaba con que pudiera \emph{caminar} por sí solo y a partir de esta base, buscar apoyo adicional para incrementar la cobertura de becas, por ejemplo.

Por tanto, para demostrar la viabilidad de este proyecto se precisa de otros instrumentos; entre ellos destaca el análisis costo-beneficio para proyectos sociales \citep{Conant2004}.

El análisis costo-beneficio para proyectos sociales establece un criterio según el cual, el beneficio social debe ser mayor que los costos en que se incurre para realizarlo.

Siguiendo la teoría general de sistemas, ya no se considera exclusivamente el ámbito reducido de la empresa, sino el sistema más amplio que es la comunidad.

Como se señaló en la introducción (página \pageref{sub:sub:Beneficio:Social}), se desea alcanzar la meta de 40\% de becas, y de ser posible, aumentarlo. Esto significa, que si se atendiera exclusivamente al tema económico, se estaría beneficiando a más de 400 estudiantes (más de 100 al año).

A esto habrá que añadir los beneficios indirectos que inciden en el total del alumnado: la educación en valores y la preparación técnica. La educación en valores incide en una mayor estabilidad familiar y social, mientras que la preparación técnica colabora en la disminución del desempleo y en la disponibilidad de personal calificado.

Por tanto, los beneficios que se están obteniendo, como reacción en cadena, habría que confrontarlos, mediante el costo, con los esfuerzos que hacen los gobiernos, empresas y sociedad para combatir el rezago educativo, prevenir la delincuencia, disminuir la violencia intrafamiliar y la desintegración familiar. ¿cuánto presupuesto se eroga para promover el empleo? ¿cuántas inversiones se pierden porque los proveedores locales no están calificados? Estas y otras preguntas forman parte de la ecuación que cuantifica el beneficio.

\section*{Acciones}

Realizar el análisis formal costo-beneficio rebasa el alcance del presente trabajo. Dada su necesidad, se apunta como la primera acción a realizar, pues determinará el atractivo del proyecto ante los diferentes grupos de interés.

Adicionalmente se pueden emprender diversos caminos que lo hagan más rentable: elegir un crecimiento más modesto o buscar apalancamiento con proveedores pueden ser opciones interesantes a condición de que no pretendan reemplazar el supuesto central de que se trata de una institución de interés social cuya viabilidad se mide por el costo-beneficio de la comunidad.

Convendría, además, extender el análisis a un horizonte de 10 años; pues los diferentes estados financieros, lo mismo que los indicadores mostraron una tendencia del proyecto a mejorar en el largo plazo.

Cronológicamente, la primera acción es la aprobación por parte de la inspectoría e inmediatamente después, la reelaboración del plan con información de campo más detallada y los instrumentos de evaluación adecuados.

\section*{Epílogo}

Para que esta escuela sea una realidad depende absolutamente de la buena voluntad de muchos actores que, actuando individual o colectivamente, materialicen las acciones aquí presentadas y las que están pendientes de realizar.

Con frecuencia se señala la economía y el mundo de la empresa como un ambiente en el que la competencia feroz hace sentir a todos en un estado de sitio, las empresas y los empresarios con frecuencia viven en angustia ante la amenaza de que un competidor más fuerte o más astuto les \emph{robe} sus ganancias.

Esta neurosis conduce a la tensión más radical en que los individuos y las sociedades pueden hallarse: la tensión entre la ética y los resultados. Es lugar común que la economía es fría y en ella no tiene cabida la moral, sólo los resultados.

El hombre pragmático, el que substituye el bien por el resultado tangible, es celebrado como triunfador y señalado como realista, con los pies en la tierra.

La más reciente crisis global de la que todavía sentimos sus efectos nos debe llevar a cuestionar de fondo este supuesto; la crisis se dio por el abuso de confianza, por olvidar la ética y defraudar al otro. Esta crisis financiera global, en el fondo es una crisis de confianza.

¿Qué fue lo que movió a los causantes a abusar de la confianza de otros? Una visión en la que triunfar contra otros competidores se volvió aún más prioritario que la subsistencia de todos.

En junio de 2009, cuando los efectos de la crisis se dejaron sentir con mayor fuerza, S.S. Benedicto XVI publicó su primera encíclica social <<Charitas in Veritate>>; en ella realiza un diagnóstico integral de la sociedad actual y propone un cambio de paradigma en la economía: la economía del <<don>>; transcribo algunas de sus líneas:

\begin{quote}
<<\textit{La caridad en la verdad pone al hombre ante la sorprendente experiencia del don. La gratuidad está en su vida de muchas maneras, aunque frecuentemente pasa desapercibida debido a una visión que antepone a todo la productividad y la utilidad... A veces, el hombre moderno tiene la errónea convicción de ser el único autor de sí mismo, de su vida y de la sociedad... La lógica del don no excluye la justicia ni se yuxtapone a ella como un añadido externo en un segundo momento... el desarrollo económico, social y político necesita, si quiere ser auténticamente humano, dar espacio al principio de gratuidad como principio de fraternidad... La economía globalizada parece privilegiar la lógica del intercambio contractual, pero directa o indirectamente demuestra que necesita la lógica de la política y la del don sin contrapartida}>> \citep{BenedictoXVI2009}.
\end{quote}
